% ************************** Thesis Abstract *****************************
% Use `abstract' as an option in the document class to print only the titlepage and the abstract.
\begin{abstract}
Aerosols moderate the Earth’s atmospheric radiation budget through scattering and absorption of solar radiation, serve as the nucleating particles for cloud formation, and their properties modulate cloud properties such as brightness, reflectivity and lifetime. To assess the effects of aerosol on climate, global climate models are an important tool. Results from the sixth phase of the Coupled Model Intercomparison Project (CMIP6) indicated that, despite participating models’ increasing sophistication, they did not all perform well in predicting surface temperature, consistently underpredicting surface temperature between 1950–1990. Published work indicates that this anomalous cooling is related to the modelled aerosol amount in the atmosphere.

This research aims to investigate sulfate aerosol formation at the process level and its role in anomalous cooling periods using a chemistry-climate model, the UK Earth System Model (UKESM1). In this model, sulfate aerosols are formed from atmospheric sulfur dioxide (\ce{SO2}) via oxidation by gas-phase hydroxyl radicals (OH), and liquid-phase ozone (\ce{O3}) and hydrogen peroxide (\ce{H2O2}). Using the transient historical prescribed sea-surface temperature simulation UKESM1 provided for the Aerosol Chemistry Model Intercomparison Project (AerChemMIP), this research shows that, although oxidant levels have a small effect on the rate of \ce{SO2} oxidation, the variation in oxidation across the historical period is driven largely by changes in \ce{SO2} emission. It is shown that during high-emission periods, OH is the dominant oxidizing channel and the main loss mechanism is dry deposition. The European region is found to exhibit high \ce{O3} oxidation which is not observed in the Eastern Asia region.  Across the historical period, the OH oxidation channel is the most sensitive to oxidant changes, followed by \ce{H2O2} oxidation while \ce{O3} oxidation is the least, indeed minimally, sensitive. A regional budget analysis shows that over the recent historical period \ce{SO2} emission migrates equatorward and eastward away from the European region, and the implications of this shift for the sulfur budget are quantified. Oxidants are also shown to impact sulfate aerosol size distribution.

\end{abstract}

\chapter{Regional aerosol formation in the CMIP6 historical period}
\label{ch5:title}
% **************************** Define Graphics Path **************************
\ifpdf
    \graphicspath{{Chapter5/Figs/Raster/}{Chapter3/Figs/PDF/}{Chapter5/Figs/}}
\else
    \graphicspath{{Chapter5/Figs/Vector/}{Chapter5/Figs/}}
\fi

% Multiple model results indicate that the global fast precipitation response to regional aerosol forcing scales with global atmospheric absorption \citep{myhrePDRMIPPrecipitationDriver2017}.

% While the emission region does not impact the scale of temperature response, it may impact the aerosol loading and indirectly control the strength of aerosol forcing.

\section*{Abstracts}
This section focuses on regional aspects of aerosol formation. It deals with both annual and seasonal trends. The region of interest includes Northeastern America, Europe, Eastern Asia and South Asia.

NEA and EUR share similar emission trends: SO2 emissions peaked around 1980 and declined thereafter. Whereas emissions from EAS and SAS are on the increase in the near present. 

\section{Introduction}
A modelling study shows that the production of oxidants over different regions may be different \cite{zhangTroposphericOzoneChange2016}. Another model study evaluates how black carbon aerosol emission at different locations affects ERF differently \citep{williamsStrongControlEffective2022}

Furthermore, geographic location can substantially influence the cooling potential of a given aerosol emission \citep{persadDivergentGlobalscaleTemperature2018}. Relative climate effects of combined sulfate, black carbon, and organic carbon aerosol emissions equivalent to China's total annual emission in 2000 result in different cooling potential.

In addition to the temperature impact, which seems to be homogeneous irrespective of the emission region. Multiple model results indicate that the global fast precipitation response to regional aerosol forcing scales with global atmospheric absorption \citep{myhrePDRMIPPrecipitationDriver2017}

This means that the emission region does not impact the scale of temperature response. However, the region of emission may impact the aerosol loading.

However, this study investigates aerosol, not emissions, and here we are interested in the oxidation potential in different regions. We focus on regions where the change in emissions has been the largest: North America, Europe, India, and China

Source attribution of sulfate in the historical period in CMIP6 simulation by source tagging: https://acp.copernicus.org/articles/17/8903/2017/ EAS is the largest contributor of emission at 31\%, followed by SAS at 11\%. Source efficiency is also reported: EUR and NEA are the most efficient at converting SO2 to sulfate in SON and DJF, but not so much in JJA 


% % From AAmas https://acp.copernicus.org/articles/16/7451/2016/
% Emissions metrics have normally been calculated for
% global emissions. However, for SLCFs, due to their short
% lifetimes compared to large-scale atmospheric mixing times,
% and because the chemistry and radiative effects on climate
% depends on the regional physical conditions, even the global
% mean radiative forcing depends on the region of emissions
% (Fuglestvedt et al., 1999; Wild et al., 2001; e.g., Berntsen et
% al., 2005; Naik et al., 2005). Then, the emission metric val-
% ues will vary for different emission locations (Fuglestvedt
% et al., 2010). In addition, distinct patterns in the tempera-
% ture response appear from all forcings (Boer and Yu, 2003;
% Shindell et al., 2010). A growing literature investigates how
% the weights of the emission metrics change as emissions
% from different regions of the world are considered. Collins
% et al. (2013) assessed variations in emission metrics for four
% different regions (East Asia, Europe, North America, and
% South Asia) for aerosols and ozone precursors, based on
% radiative forcings from consistent multimodel experiments
% from the Hemispheric Transport of Air Pollution (HTAP) ex-
% periments given by Yu et al. (2013) and Fry et al. (2012).
% Collins et al. (2010) also investigated how emission met-
% ric values differ between regions, including vegetation re-
% sponses. Bond et al. (2011) quantified differences in RFs for
% BC and OC emissions from different locations and types of
% emissions

\subsection{\texorpdfstring{\ce{SO2}}{SO2} emissions trends over different regions}

Zonal asymmetry of anthropogenic aerosol forcing in the recent decades \citep{diaoAnthropogenicAerosolEffects2021}


Explain the emissions trends over ENA, EU, EAS, SAS. 

The distribution of aerosol by each source is explored in \citet{yangGlobalSourceAttribution2017}. Sulfate aerosols can be transported further from the source area.

\subsection{Regional emission changes on ERF}

cite \citet{kalisorasDecomposingEffectiveRadiative2024}

\subsection{ERF analysis with sensitivity to aerosol AOD}

\subsection{Research gaps/questions}

\citet{bellouinRegionalSeasonalRadiative2016} uses older data to diagnose summer- and winter-specific erf for different regions. Good, and I have used that in chapter 4---still no sensitivity analysis and no CMIP. \citet{bellouinBoundingGlobalAerosol2020} is a good reference on aerosol erf sensitivity, which I cited. 


\citet{kalisorasDecomposingEffectiveRadiative2024} conducts a multi-model intercomparison of piClims and histSST. Include aerosol AOD from the ensemble, too. But the work does not extend to aerosol production from chemical mechanisms, nor does it attempt to diagnose model sensitivity. That's where I come in.


\section{Methodology}

\subsection{Theoretical framework for ERF analysis with sensitivity to precursor emissions}

Overview
\begin{equation} \label{ch5:eq:flow-of-response}
\begin{aligned}
    P_{\ce{SO2}} 
    \xRightarrow{\gamma_1} Q_{\ce{SO2}}
    \xRightarrow{\gamma_2} P_{S(VI)}
    \xRightarrow{\gamma_3} Q_{S(VI)}
    & \xRightarrow{\gamma_{4a}} \tau_{a}
    & \xRightarrow{\gamma_{5a}} ERF\\
    Q_{S(VI)}
    & \xRightarrow{\gamma_{4b}} N_{50}
    \xRightarrow{\gamma_{5b}} N_{d}
    \xRightarrow{\gamma_{6b}} \tau_{c}
    & \xRightarrow{\gamma_{7b}} ERF
\end{aligned}
\end{equation}


cite
\citet{bellouinBoundingGlobalAerosol2020}

\subsubsection*{Sulfate burden as a function of \ce{SO2} emission}

The change in the material balance of \ce{SO2} over time can be written as the sum of production and removal tendencies.
\begin{align}
    \dv{Q_{\ce{SO2}}}{t} 
    & = \sum_i P_{\ce{SO2}}^i - \sum_j R_{\ce{SO2}}^j  \\
    &= P_{\ce{SO2}}^n + P_{\ce{SO2}}^a + P_{\ce{SO2}}^c 
    - R_{\ce{SO2}}^d - R_{\ce{SO2}}^w - R_{\ce{SO2}}^t - R_{\ce{SO2}}^c 
\end{align}

Where, $P_{\ce{SO2}}^i$ denote production of \ce{SO2} from combination of natural ($n$), anthropogenic ($a$), and chemical ($c$) sources. Similarly, \ce{SO2} removal, $R_{\ce{SO2}}^j$, is the sum of wet ($w$) and dry ($d$) deposition, transport out of the region or air parcel, and chemical ($c$) losses.

Consider \ce{SO2} in steady state. The change in \ce{SO2} burden over time is zero.

\begin{equation}
\begin{aligned}
    0 & = \sum_i P_{\ce{SO2}}^i - \sum_j R_{\ce{SO2}}^j \\
    \sum_i P_{\ce{SO2}}^i &= \sum_j R_{\ce{SO2}}^j \\
    &= \sum_j L_{\ce{SO2}}^j Q_{\ce{SO2}} 
\end{aligned}
\end{equation}

For historical simulation and control simulation, define $\Delta X$ as the change in X, where X is a chemical or atmospheric state, due to the changes in the historical aerosol precursor emissions. $\Delta X = X_{\text{\histsst{}}}-X_{\text{\sstpiaer{}}}$. 

For instance, the total change in \ce{SO2} production due to historical anthropogenic emission is 
\begin{equation}
\begin{aligned}
    \Delta P_{\ce{SO2}}
    & = \sum_i P_{\ce{SO2}, \text{\histsst{}}}^i - \sum_i P_{\ce{SO2}, \text{\sstpiaer{}}} \\
    & = P_{\ce{SO2}, \text{\histsst{}}}^a - P_{\ce{SO2}, \text{\sstpiaer{}}}^a \\
    & = \Delta P_{\ce{SO2}}^a
\end{aligned}
\end{equation}

That is, for the change in \ce{SO2} emission between PD and PI, 
\begin{equation}
    \Delta Q_{\ce{SO2}} = \gamma_1 \Delta P_{\ce{SO2}}
\end{equation}

where $\gamma_1$ is the sensitivity of \ce{SO2} burden, and is an arithmetic sum of all the pseudo first-order loss rate constant, $\sum_j L_{\ce{SO2}}^j$, $ \Delta Q_{\ce{SO2}}$ to change in \ce{SO2} emission, $\Delta Q_{\ce{SO2}}$. $\gamma_1$ could be calculated from the linear relationship between the change in oxidant concentration due to $\Delta P_{\ce{SO2}}$ and the change in oxidant concentration due to $\Delta Q_{\ce{SO2}}$

\subsubsection*{Production of sulfate as a function of SO2 burden}

Let us consider the reaction rate of sulfate aerosol, collectively written as sulfur in the +6 oxidation state (S(VI)) as parameterised in UKESM1, as described by \citet{mannDescriptionEvaluationGLOMAPmode2010}. The production of S(VI) (in \unit{molecules~cm^{−3}~s^{−1}}), written using function notation, i.e. $f(x,y)$ is a function of $x$ and $y$, is given by

\begin{equation}
\begin{aligned}
    rate_{\ce{SO2 + OH}}([\ce{SO2}],[OH],[M],T) &= k_1[\ce{SO2}][\ce{OH}] \\
    rate_{\ce{SO2 + O3}}(F,L,[\ce{SO2}],[\ce{O3}], [\ce{H+}],[M],T) &= F \cdot \left( \dv{[S(IV)]}{t}\right)_{\ce{SO2 + O3}} \cdot L \cdot N_a \cdot \frac{1}{\rho_w}\\
    rate_{\ce{SO2 + H2O2}}(F,L,[\ce{SO2}],[\ce{H2O2}], [\ce{H+}],[M],T) &= F \cdot \left( \dv{[S(IV)]}{t}\right)_{\ce{SO2 + H2O2}} \cdot L \cdot N_a \cdot \frac{1}{\rho_w}
\end{aligned}
\end{equation}


where rate constants, $k_i$, depend on temperature, T, and air concentration, [M]. The aqueous-phase productions, $\left( \dv{[S(IV)]}{t}\right)_{\ce{SO2 + O3}}$ and $\left( \dv{[S(IV)]}{t}\right)_{\ce{SO2 + H2O2}}$,  follow Equation \ref{ch4:eq:in-cloud-sulfate-prod} and is a function of aerosol droplet pH, [\ce{H+}], T, and the gas-phase concentration of the reactants, [\ce{O3}] or [\ce{H2O2}]. $F$ is the cloud fraction; $L$ is the cloud liquid water content; $N_a$ is the Avogadro's constant; and $\rho_w$ is the density of water. In the model version used in this work, global cloud pH is set to 4.0 when the local \ce{SO2} mixing ratio exceeds 0.5 ppb, and to 5.0 otherwise. 

The total rate of chemical production of S(VI), $rate_{\text{S(VI)}}$, is a summation of each production pathway. 

\begin{equation} \label{ch5:eq:rate_s(vi)}
\begin{aligned}
    rate_{\text{S(VI)}} = & rate_{\ce{SO2 + OH}}([\ce{SO2}],[OH],[M],T) \\
    &+ rate_{\ce{SO2 + O3}}(F,L,[\ce{SO2}],[\ce{O3}], [\ce{H+}],[M],T) \\
    &+ rate_{\ce{SO2 + H2O2}}(F,L,[\ce{SO2}],[\ce{H2O2}], [\ce{H+}],[M],T)
\end{aligned}
\end{equation}
    

Reaction rate of X, $\dv{[\text{X}]}{t}$ in \unit{molecule~cm^{-3}~s^{-1}}, which is an intensive variable, can be expressed as atmospheric production tendency, in $P_\text{X}$ \unit{Tg(X)~yr^{-1}}, an extensive variable, by

\begin{align*}
    P_\text{X} & = \boxed{\dv{[\text{X}]}{t}} \frac{\unit{molecules~X}}{\unit{cm^3~s}} 
    \cdot \frac{10^6~\unit{cm^3}}{1~\unit{m^3}}
    \cdot \boxed{vol}~\unit{m^3}\\
    & \cdot \frac{1~\unit{mol~X}}{N_A~\unit{molecules~X}}
    \cdot \frac{M_\text{X}~\unit{kg(X)}}{1~\unit{mol~X}}
    \cdot \frac{10^{-9}~\unit{Tg(X)}}{1~\unit{kg(X)}}
    \cdot \frac{N_{\text{s-in-yr}}~\unit{s}}{1~\unit{year}}\\
    &= \boxed{\dv{[\text{X}]}{t}} \cdot \boxed{vol}\cdot \frac{M_{\text{X}} N_{\text{s-in-yr}}}{N_A} \cdot 10^{-3} \label{eq:p_rate_conv}
\end{align*}

Similarly, concentration of trace gas X, [X] in \unit{molecule~cm^{-3}~s^{-1}}, which is an intensive variable, can be expressed as the burden, $Q_\text{X}$ in \unit{Tg(X)}, an extensive variable, as

\begin{equation}
    Q_\text{X}= \boxed{[\text{X}]} \cdot \boxed{vol}\cdot \frac{M_{\text{X}} }{N_A} \cdot 10^{-3}
\end{equation}

Using the conversions above, Equation \ref{ch5:eq:rate_s(vi)} then can be expressed as

\begin{equation}
\begin{aligned}
    P_{\text{S(VI)}} = & P_{\ce{SO2 + OH}}(Q_{\ce{SO2}},[OH],[M],T) \\
    &+ P_{\ce{SO2 + O3}}(F,L,Q_{\ce{SO2}},[\ce{O3}], [\ce{H+}],[M],T) \\
    &+ P_{\ce{SO2 + H2O2}}(F,L,Q_{\ce{SO2}},[\ce{H2O2}], [\ce{H+}],[M],T)
\end{aligned}
\end{equation}

When quantifying the changes in reaction rates due to historical \ce{SO2} emissions in each grid box of the model by subtracting the perturbed simulation from the control simulation, many variables do not change. There is no change in surface air temperature in the atmosphere-only simulations, as the surface temperature is constrained, so changes in air temperature are negligible. Again, assume that the change in oxidant concentration due to the reactions with \ce{SO2} is negligible and [OH], [\ce{O3}], and [\ce{H2O2}] are now treated as constant. Cloud fraction, F, and liquid water content, L, are linked and cannot be linearised. Define a new internal change in cloud, $F' = F \cdot L$. 


\begin{equation}
\begin{aligned}
    P_{\text{S(VI)}} = & P_{\ce{SO2 + OH}}(Q_{\ce{SO2}}) \\
    &+ P_{\ce{SO2 + O3}}(F',Q_{\ce{SO2}}) \\
    &+ P_{\ce{SO2 + H2O2}}(F',Q_{\ce{SO2}})
\end{aligned}
\end{equation}

The total derivative of sulfate production could be written as

\begin{equation}
\begin{aligned}
    \dd{P_{\text{S(VI)}}} = & 
    \dv{P_{\ce{SO2 + OH}}}{Q_{\ce{SO2}}} \dd{Q_{\ce{SO2}}} \\
    &+\pdv{P_{\ce{SO2 + O3}}}{F'}\bigg|_{Q_{\ce{SO2}}} \dd{F'} 
     +\pdv{P_{\ce{SO2 + O3}}}{Q_{\ce{SO2}}}\bigg|_{F'} \dd{Q_{\ce{SO2}}} \\
    &+\pdv{P_{\ce{SO2 + H2O2}}}{F'}\bigg|_{Q_{\ce{SO2}}} \dd{F'} 
     +\pdv{P_{\ce{SO2 + H2O2}}}{Q_{\ce{SO2}}}\bigg|_{F'} \dd{Q_{\ce{SO2}}} \\
    % =& \left[ 
    %     \dv{P_{\ce{SO2 + OH}}}{Q_{\ce{SO2}}} 
    %     + \pdv{P_{\ce{SO2 + O3}}}{Q_{\ce{SO2}}}\bigg|_{F'} 
    %     + \pdv{P_{\ce{SO2 + H2O2}}}{Q_{\ce{SO2}}}\bigg|_{F'}
    % \right] \dd{ Q_{\ce{SO2}}} \\
    % &+ \left[
    %    \pdv{P_{\ce{SO2 + O3}}}{F'}\bigg|_{Q_{\ce{SO2}}} 
    %     +  \pdv{P_{\ce{SO2 + H2O2}}}{F'}\bigg|_{Q_{\ce{SO2}}} \right] \dd{F'} \\
\end{aligned}
\end{equation}

This derivation assumes that all variables are independent.

To take into account the aerosol-cloud interaction and its effects on cloud properties, $\dd{F'}$ could be written as a composite function which is expressed as a chained derivation as follows, assuming that \ce{SO2} burden is the only perturbation.

\begin{equation}
    \dd{F'} = \pdv{F'}{N_d} \pdv{N_d}{N_{50}} \pdv{N_{50}}{Q_{\text{S(VI)}}} \pdv{Q_{\text{S(VI)}}}{P_{\text{S(VI)}}} \pdv{P_{\text{S(VI)}}}{Q_{\ce{SO2}}} \dd{Q_{\ce{SO2}}}
\end{equation}

In full, 
\begin{align}
    \Delta{P_{\text{S(VI)}}} 
    = & \gamma_{2,\ce{OH}} \Delta{Q_{\ce{SO2}}}\\ 
      & + (\gamma_{2, \ce{O3}}^{direct} + \gamma_{2, \ce{O3}}^{indirect}) \Delta{Q_{\ce{SO2}}} \\ 
      & + (\gamma_{2, \ce{H2O2}}^{direct} + \gamma_{2, \ce{H2O2}}^{indirect}) \Delta{Q_{\ce{SO2}}} \\
    = & (\gamma_{2,\ce{OH}} + \gamma_{2,\ce{O3}} + \gamma_{2,\ce{H2O2}}) \Delta{Q_{\ce{SO2}}} \\
     \Delta {P_{\text{S(VI)}}} = & \gamma_2 \Delta{Q_{\ce{SO2}}} 
\end{align}


Again, $\gamma_{2}$ and its component from each production pathway could be calculated from the linear relationship between oxidation tendency and burden. 

\subsubsection{sulfate aerosol production on the sulfate burden}

Next, consider the impact of sulfate aerosol production on the sulfate burden. The derivation is the same as the production of \ce{SO2} burden from production. 
 
\begin{equation}
    \Delta Q_{\text{S(VI)}} = \gamma_3 \Delta P_{\text{S(VI)}}
\end{equation}

\subsubsection{sulfate aerosol production on the sulfate burden}

Aerosol optical depth, $\tau_a$, is a measurable aerosol property that is shown to contribute towards the aerosol direct effect. Aerosol optical depth is the radiative property in which aerosol of all types is used. The total aerosol loading, $Q_{\text{tot}}$, is the sum of all aerosol burdens. 

\begin{equation}
\begin{aligned}
    \dd{\tau_a(Q_{\text{S(VI)}}, Q_{\text{BC}}, Q_{\text{OC}})} =& \dv{\tau_a}{Q_{\text{tot}}} \bigg\{\pdv{\tau_a}{Q_{\text{S(VI)}}}\bigg|_{Q_{\text{BC}},Q_{\text{OC}}}\dd{Q_{\text{S(VI)}}} \\
    & + \pdv{\tau_a}{Q_{\text{BC}}}\bigg|_{Q_{\text{S(VI)}},Q_{\text{OC}}}\dd{Q_{\text{OC}}} \\
    & + \pdv{\tau_a}{Q_{\text{OC}}}\bigg|_{Q_{\text{S(VI)}},Q_{\text{BC}}}\dd{Q_{\text{BC}}}\bigg\} \\
    = & \dv{\tau_a}{Q_{\text{tot}}} \{\dd{Q_{\text{S(VI)}}} + \dd{Q_{\text{OC}}} + \dd{Q_{\text{BC}}}\}
\end{aligned}
\end{equation}

The aerosol precursor includes OC and BC emissions, and the equation reflects that. However, as shown in Chapter \ref{ch3:title}, aerosol burden is primarily from SO2, so the estimation below could be made.

\begin{equation}
    \Delta \tau_a \approx \gamma_{4a}\Delta Q_{\text{S(VI)}}
\end{equation}

This connects burden to the theoretical framework in \citet{bellouinAerosolForcingClimate2011}, which analyses the impact of forcing due to changes in aerosol optical depths and aerosol-induced cloud on radiative effects. 

\subsubsection{Aerosol radiation interaction and aerosol optical depth}

The change in radiative effect due to aerosol forcing could be written as

\begin{equation}
\begin{aligned}
    F   &= \Delta R = ARI + ACI + RA \\
    ARI &= \frac{\partial R}{\partial \tau_a} \Delta \tau_a \\
    ACI &= \frac{dR}{ln N_d} \Delta ln N_d \\
    RA  &= \frac{\partial R}{\partial R_{atm}} \frac{d R_{atm}}{d \tau_a} \Delta \tau_a 
\end{aligned}
\end{equation}


Where F is ERF which is the sum of aerosol-radiation interaction (ARI), aerosol-cloud interaction (ACI) + rapid adjustment (RA)

Let us consider each of the radiative effects.


\subsection{Defintion of regions}

\begin{figure}[ht!]
    \centering
    \includegraphics{Chapter5/Figs/blank_ipcc_region_map_all_regions.png}
    \caption{Definition of domain boundary following the IPCC report. Five land regions: East North America (ENA), North Europe (NEU), West and Central Europe (WCE), South Asia (SAS), and East Asia (EAS) are noted for their usage throughout the chapter. NEU and WEU are grouped in this report and referred to as the European region (EU).}
    \label{fig:ch5:region-def}
\end{figure}

This report adopts the definition of land regions from the IPCC report. Figure \ref{fig:ch5:region-def} shows the domain boundaries of the regions of interest. East North America (ENA), North Europe (NEU), and West and Central Europe were chosen for their early increase in emissions. South Asia (SAS) and East Asia (EAS) are selected for their characteristic recent growth in emissions. For simplicity, North Europe and West and Central Europe are combined to summarise the impacts from Europe. This definition does not include the Mediterranean and Eastern Europe for simplicity only. The Mediterranean region is excluded from this report as it is considered a mixed land and ocean region. Eastern Europe is excluded due to their less clear emission trend, and to limit the bounded area to facilitate cross-comparison with the other three regions.


\begin{figure}[ht!]
    \centering
    \includegraphics{Chapter5/Figs/so2_series_histsst.png}
    \caption{a) Global and regional time series of changes in \ce{SO2} emissions due to aerosol precursor emissions. b-c) Global distribution of change in \ce{SO2} emission due to aerosol precursor emissions between 1960--1990 and 2000--2015.}
    \label{fig:ch5:emission_series}
\end{figure}

Figure \ref{fig:ch5:emission_series} shows the changes in \ce{SO2} emissions due to aerosol precursors (\histsst{} minus \sstpiaer{}). The global increase in the historical \ce{SO2} emissions increases to \qty{80}{Tg(S)~yr^{-1}} in 1975 and remained above \qty{65}{Tg(S)~yr^{-1}}. Europe and East North America drive the initial rise. When emissions are restricted in these regions, the high \ce{SO2} emissions are sustained by the rapid growth from East and South Asia. This paints a picture of the divergence in the four selected regions.

% \begin{figure}[ht!]
%     \centering
%     \includegraphics[width=\linewidth]{Chapter5/Figs/emiso2_seasonal_regional_histsst.png}
%     \caption{Climatology of \ce{SO2} emissions for four regions of interest for two periods.}
%     \label{fig:ch5:emission_seasonal}
% \end{figure}

\section{Results and discussions}

This chapter investigates the regional characteristics of sulfate aerosol formation and the impact of emission location on radiative forcing. Similar to the other chapters, this chapter examines regional trends in aerosol precursor emissions and aerosol formation, and their effects on cloud properties and climate. All analyses show the change due to aerosol precursor emissions, i.e., the result of subtracting \sstpiaer{} from \histsst{} unless noted otherwise. As a connection to Chapter \ref{ch4:title}, the anomalous cooling (1960--1990) and near-present (2000-2015) periods are chosen for analysis of global distribution and seasonal variability. Then, a new analysis of process sensitivity by region is discussed. 

\subsection{Regional difference of OH, \ce{O3}, \ce{H2O2}, and \ce{SO2}}

\begin{figure}[ht!]
    \centering
    \includegraphics[width=\linewidth]{Chapter5/Figs/ox_conc_annual_mean_histsst_surface_to_5km.png}
    \caption{Annual mean of historical oxidant concentration from the surface up to 5 km. The concentrations are derived from \histsst{} simulation, showing the historical concentration level, not the change due to historical aerosol precursor emissions. Note the difference in the y-axis scale across the subfigures. }
    \label{fig:ch5:oxidant_series}
\end{figure}


Location may influence aerosol formation by altering the concentrations of reactants for sulfate aerosol formation. Figure \ref{fig:ch5:oxidant_series} shows the annual mean historical concentration of the oxidants from the surface up to 5 km altitude. OH concentrations rise over the historical period for all regions. The baseline level of OH in South Asia is double that in other areas due to its lower latitude, which promotes \ce{O3} photolysis and produces \ce{O(1D)}, which can react with \ce{H2O} to form OH. OH concentrations over East and South Asia steadily increase while those over Europe and Eastern North America plateau. This is attributable to the trends of \ce{O3} precursors, which promote \ce{O3} concentration, and hence OH.

\ce{O3} concentration increases globally and regionally over the historical period. The average concentration is between \num{6.0e11} and \qty{7.5e12}{cm^{-3}} in 1850 and rises to approximately \qty{1e12}{cm^{-3}} for all regions. The \ce{O3} trends and magnitude in all regions are comparable.

Location plays a role in determining the baseline concentration level of \ce{H2O2}. Figure \ref{fig:ch5:oxidant_series}c) shows that \ce{H2O2} concentration over Europe is the lowest, at \qty{4.2}{cm^{-3}} at the beginning of the simulation.  Meanwhile, \ce{H2O2} concentration over South Asia is three times greater than in Europe during the same time. This is likely due to higher photochemical activity in lower-latitude regions, as the baseline level is highest in South Asia, which lies at the lowest latitude.  Over the historical period, \ce{H2O2} concentration increases in all regions, but the increment is less in Europe and Eastern North America than in the other Asian regions. 

Finally, \ce{SO2} concentration shows the most unique features amongst all the tracers. The regional trends in \ce{SO2} concentration follow that of emissions. It could be seen that \ce{SO2} concentration in Europe and East North America increases to a maximum between 1960 and 1975, then decreases afterwards, while the increases in East and South Asia start in 1950 and continue to the present.

In short, the precursor gases to sulfate aerosol formation vary widely between regions, depending on the emission profile, in the case of \ce{SO2}, as well as the photochemical activities over the areas, which determine the photolysis rate of \ce{O3}, OH and \ce{H2O2}.

% \begin{figure}[ht!]
%     \centering
%     \includegraphics[width=\linewidth]{Chapter5/Figs/ox_conc_monthly_mean_histsst_surface_to_5km.png}
%     \caption{Climatology of historical oxidant concentration from the surface up to 5 km for four regions of interest for two periods.}
%     \label{fig:ch5:emission_seasonal}
% \end{figure}



\subsection{Regional difference in historical sulfate aerosol production}

Secondary sulfate aerosols form via either gas- or aqueous-phase oxidation of \ce{SO2}. In gas-phase oxidation with OH, the reaction tendency depends directly on gas concentration and the reaction rate constant. Aqueous-phase oxidation involves more factors, including cloud liquid water content. Like the gaseous precursors, cloud properties vary significantly across regions and further complicate the determination of the tendency toward sulfate aerosol production. 

\begin{figure}[ht!]
    \centering
    \includegraphics[width=\linewidth]{Chapter5/Figs/regional_cl_times_lwc_series_histsst_sstpiaer_histsst_only.png}
    \caption{a) Annual mean of products of cloud fraction (cf) and cloud liquid water content (lwc) below 10 km in the historical period from \histsst{}. b-c) Global distribution of the products of cf and lwc averaged between 1960--1990 and 2000--2015. }
    \label{fig:ch5:cf_times_lwc_series}
\end{figure}

Cloud fraction and liquid water content are featured in aqueous-phase oxidation and determine the available soluble liquid in a simulation grid, as described in Section \ref{ch1:so2-oxidation}. The products of the two parameters, termed $cf \times lwc$, are another factor that influences sulfate aerosol formation. Figure \ref{fig:ch5:cf_times_lwc_series} shows the annual mean and global distribution from the surface up to 10 km. Global mean $cf \times lwc$ is constant over the historical period at \qty{0.025}{\gram\per\cubic\metre}. South Asia has the lowest $cf\times lwc$ at 50\% that of the global average, while Europe is the cloudiest of the four regions.


\begin{figure}[ht!]
    \centering
    \includegraphics[width=\linewidth]{Chapter5/Figs/regional_oxidation_series_histsst_sstpiaer_diff.png}
    \caption{Annual mean of total oxidation tendency over a) Europe, b) East North America, c) East Asia, and d) South Asia due to changes in the historical aerosol precursor emissions (\histsst{} minus \sstpiaer{}).}
    \label{fig:ch5:oxidation_series}
\end{figure}


Figure \ref{fig:ch5:oxidation_series} shows sulfate formation tendency from \ce{SO2} reaction with OH, \ce{O3} and \ce{H2O2} due to changes in aerosol precursor emissions over the historical period in Europe, East North America, East Asia, and South Asia. \ce{SO2} oxidation trends in Europe follow that of \ce{SO2} emissions, with rapid rises since the beginning of the simulation to 1975, followed by a rapid decrease.  Amongst the three pathways, \ce{SO2 + O3} shows the greatest increase at \qty{2.1}{Tg(S)~yr^{-1}} during the maximum production in 1975. Despite \ce{SO2 + H2O2} being an aqueous-phase reaction, it does not share the same trend as \ce{SO2 + O3}. Compared to the global average and to other regions, high cloud liquid water over Europe facilitates aqueous-phase oxidation (Figure \ref{fig:ch5:cf_times_lwc_series}), making \ce{H2O2} concentration the limitation of reaction rate, and not cloud properties. 

Increases in \ce{SO2} emission cause \ce{SO2} oxidation in East North America to gradually rise until the oxidation tendencies reach the maximum in 1975, similar to Europe. Unlike Europe, the total oxidation is aligned almost equally between the three pathways, with gas-phase reaction slightly above the other pathways from 1960 onwards. The higher proportion of \ce{SO2 + OH} oxidation is due to the lower $cf\times lwc$ in the region, combined with a higher OH concentration than that of Europe. 

Figure \ref{fig:ch5:oxidation_series}c) shows sharp increases of \ce{SO2} oxidation in East Asia, with gas-phase oxidation with OH as the primary pathway. East Asia has the same latitude band as East North America and shares similar cloud profiles. Their main difference lies in the evolution of \ce{SO2} emissions over the historical period. 

South Asia region sees a sharp rise in \ce{SO2} oxidation post-1950. The main effect is observed in the gas-phase oxidation, which rose to \qty{2.3}{Tg(S)~yr^{-1}} in 2015. Despite high \ce{O3} and \ce{H2O2} concentration and high \ce{SO2} emission over the region, changes in aqueous-phase oxidation are limited. South Asia has a characteristic low $cf \times lwc$, which constrains aqueous-phase oxidation and favours gas-phase oxidation. 

Comparing the four regions, Europe's and Eastern North America's early increases and emission restrictions led to a peak in \ce{SO2} oxidation in the 1970s. However, the two regions exhibit different oxidation activities due to the difference in background pollution and cloud abundance. \ce{SO2 + O3} is the most important reaction in Europe, while the increases in oxidation over East North America do not show a clear dominance. East North America and East Asia may have different emission trends, but their oxidation features are similar. All \ce{SO2} oxidation reactions contribute nearly equally towards sulfate production. South Asia, on the other hand, shows a characteristically high gas-phase oxidation rate and a minimal \ce{SO2 + O3} tendency, a stark contrast to Europe.


\begin{figure}[ht!]
    \centering
    \includegraphics[width=\linewidth]{Chapter5/Figs/regional_tot_so2_oxi_series_histsst_sstpiaer.png}
    \caption{a) Global and regional time series of changes in total \ce{SO2} oxidation due to aerosol precursor emissions (\histsst{} minus \sstpiaer{}). b-c) Global distribution of changes in total \ce{SO2} oxidation due to aerosol precursor emissions between 1960--1990 and 2000--2015. }
    \label{fig:ch5:tot_oxidation_series}
\end{figure}

The combined oxidation shown in Figure \ref{fig:ch5:tot_oxidation_series} illustrates the trends and location of oxidation. Total \ce{SO2} oxidation over Europe makes the most significant share at its peak production in 1975, contributing \qty{4.48}{Tg(S)~yr^{-1}} or 11\& of to the global total secondary production. As East Asian emissions ramp up, their total oxidation overtakes Europe in 1990 and adds \qty{6.68}{Tg(S)~yr^{-1}} to the global total. Global distribution of oxidation shows that sulfate aerosols are formed near the emission location (Figure \ref{fig:ch5:emission_series}) and that most of the production is confined within the defined region boundary.



% \begin{figure}[ht!]
%     \centering
%     \includegraphics[width=\linewidth]{Chapter5/Figs/regional_oxidation_seasonal_histsst_sstpiaer_diff.png}
%     \caption{Climatology of \ce{SO2} emissions for four regions of interest for two periods.}
%     \label{fig:ch5:emission_seasonal}
% \end{figure}

% \begin{itemize}
%     \item there is seasonality in all regions, of varying degrees.
%     \item \ce{SO2} oxidation over Europe shows that in winter, most \ce{SO2} is oxidised with \ce{O3} while both \ce{OH} and \ce{H2O2} are more important in summer.
%     \item \ce{SO2} oxidation in the East of North America shows a similar profile to Europe, but with less contribution from \ce{O3} in winter, reducing the total oxidation in winter.
%     \item East Asia in 2000-2015 shows a strong summer oxidation by OH and \ce{H2O2}.
%     \item South Asia shows the least seasonal variation, with \ce{OH} as the main oxidation pathway throughout the year, followed by \ce{H2O2} 
% \end{itemize}

In short, the conditions at the \ce{SO2} emission source play an essential role in determining the oxidation tendency, which shows a great variety across the selected regions. Europe features high cloud cover and low OH levels, so aqueous-phase oxidation dominates, while South Asia, which has minimal cloud cover, tends towards more gas-phase oxidation. That is, the location of the emission dictates the atmospheric parameters and tracers available for the reaction to occur, leading to a drastic difference in sulfate formation.


\subsection{Regional difference in historical aerosol properties}


GLOMAP-mode, the aerosol microphysical scheme used in UKESM1, simulates aerosol number and mass observation independently, allowing the aerosol formation pathway to influence its properties. The important things to look at are aerosol mass loading and particle number concentration. These two show the microscopic properties of aerosols. Aerosols with diameters greater than 50 nm (N50) are relevant to cloud formation as they activate cloud droplet condensation. 


\begin{figure}[ht!]
    \centering
    \includegraphics[width=\linewidth]{Chapter5/Figs/regional_mmrso4_series_histsst_sstpiaer.png}
    \caption{a) Global and regional time series of changes in tropospheric sulfate aerosol burden due to aerosol precursor emissions (\histsst{} minus \sstpiaer{}). Readers are to refer to Figure \ref{fig:ch3:s-budget} for the full global trend. b-c) Global distribution of changes in total sulfate aerosol burden due to aerosol precursor emissions between 1960--1990 and 2000--2015.}
    \label{fig:ch5:so4_series}
\end{figure}


Sulfate aerosol is the direct product of \ce{SO2} oxidation. Figure \ref{fig:ch5:so4_series} shows the global and regional changes in sulfate aerosol burden due to increased historical aerosol precursor emissions. All oxidation reactions produce sulfate aerosol mass, so there is a direct relationship to total oxidation in Figure \ref{fig:ch5:tot_oxidation_series}. For instance, the model simulates an increase of \qty{0.025}{Tg(S)} in sulfate aerosol burden at the peak of production in Europe. The global distribution of sulfate aerosol burden is more dispersed than its production tendency, and is not captured within the region boundaries. This is one of the limitations of using a static regional boundary to define the regional effect of emission location. However, in this case of sulfate burden, a large proportion of aerosols remains within the region's borders. Hence, the regional trends in sulfate burden remain valid and reflect the characteristics of their emissions. For example, aerosol produced in the defined European region is transported eastward out of the bounded area, yet the trends in Figure \ref{fig:ch5:so4_series}a still follow that of Figure \ref{fig:ch5:tot_oxidation_series}a.


\begin{figure}[ht!]
    \centering
    \includegraphics[width=\linewidth]{Chapter5/Figs/regional_od550aer_series_histsst_sstpiaer.png}
    \caption{a) Global and regional time series of changes in aerosol optical depth (AOD) at 550 nm due to aerosol precursor emissions (\histsst{} minus \sstpiaer{}). b-c) Global distribution of changes in AOD due to aerosol precursor emissions between 1960--1990 and 2000--2015.}
    \label{fig:ch5:aod_series}
\end{figure}


% \begin{figure}[ht!]
%     \centering
%     \includegraphics[width=\linewidth]{Chapter5/Figs/regional_od550aer_seasons_histsst_sstpiaer.png}
%     \caption{Caption}
%     \label{fig:placeholder}
% \end{figure}


Aerosols are typically 10-100 nm in size, making them efficient at scattering short-wave radiation via Mie scattering. The impacts of historical change in aerosol precursor emissions on aerosol optical depth at 550 nm are shown in Figure \ref{fig:ch5:aod_series}. Global mean aerosol optical depth, discussed in Chapter \ref{ch3:title}, shows a steep increase starting from 1950, reaching 0.42 in 2015. The regional mean shows the heterogeneity and local variation in the change. AOD above Europe and East North America is similar in both the magnitude and evolution, with a rapid increase from 1875 to 1975 and a steep decline thereafter. East and South Asia also show similar trends, with increased AOD up to the end of 2014. One notable feature is that the AOD above Asia is 50\% greater than that of the peak of Europe and East North America (0.31 compared to 0.18). In South Asia, this could be due to higher black and organic carbon emissions, as shown in Figure \ref{fig:app2:emioa_series}--\ref{fig:app2:emibvoc_series} of Appendix \ref{app2:title}.


To distinguish the effects of each aerosol precursor component, timeslice simulations with each component are analysed and compared with AOD from the \histsst{} simulation. The timeslice simulations, \textit{piClim-X}, where X is a climate forcing agent, are designed to quantify the radiative effects of that agent. Pre-industrial surface temperatures are constrained to pre-industrial values, and a single forcing agent is used, with the emission or concentration from the present day (2014) set for each simulation. In AerChemMIP simulation, aerosol forcing (\textit{aer}) is the combination of organic carbon (\textit{OC}), black carbon (\textit{BC}) aerosol emissions and \ce{\textit{SO2}} emissions. This timeslice experiment is estimated as the breakdown of the last year of \histsst{} simulation.

\begin{figure}[ht!]
    \centering
    \includegraphics[width=\linewidth]{Chapter5/Figs/aod_map_piClim.png}
    \caption{Global distribution of aerosol optical depth at 550 nm from 30-year mean of timeslice simulations. Plots show difference between \textit{piClim-control} and a) \textit{piClim-OC}, b) \textit{piClim-BC}, c) \textit{piClim-SO2}, and d) \textit{piClim-aer}. Colourbar ranges in figure a-b) are 25\% of figure c-d).}
    \label{fig:ch5:aod_map_piClim}
\end{figure}

Figure \ref{fig:ch5:aod_map_piClim} shows the impacts of present-day emissions of each aerosol component and the total aerosol precursor on AOD. The change in AOD is calculated by subtracting \textit{piClim-control} from \textit{piClim-X}, with X being \textit{OC}, \textit{BC}, \textit{SO2} and \textit{aer}, using the last 30 years of the 45-year timeslice simulation. OC aerosol emissions increase AOD in Canada, Southeast Asia, the Amazon forest, and Siberia, all of which are outside the boundary of the regions of interest of this Chapter. Black carbon aerosol emissions are shown to increase AOD in both East and South Asia. \textit{piClim-SO2} is the primary contributor to the total change in AOD in Figure \ref{fig:ch5:aod_map_piClim}d. 

\begin{figure}[ht!]
    \centering
    \includegraphics[width=\linewidth]{Chapter5/Figs/piClim-X-aod-component.png}
    \caption{Regional change in aerosol AOD at 550 nm from timeslice simulations and transient fixed-SST simulation. The changes in aerosol AOD from the difference between AOD from  \textit{piClim-X} minus \textit{piClim-control} using the last 30 years' output, and from \histsst{} minus \sstpiaer{} for the transient simulation using the last five years' output (2005--2015). The error bars show one standard deviation from the regional mean.}
    \label{fig:ch5:piClim-aod-component}
\end{figure}

To further quantify the regional impact of each component of aerosol precursor emissions, Figure \ref{fig:ch5:piClim-aod-component} shows regional and global mean changes in AOD due to each aerosol component from timeslice simulations and from the last 5 years of \histsst{} minus \sstpiaer{} simulation (2005--2015). \ce{SO2} is the main aerosol component that changes the AOD. A similar conclusion could be drawn for the East North America region. In East Asia, black carbon aerosol emissions contributed \num[]{0.02+-0.05}, or 10\%, to the increase in AOD, amounting to \num[]{0.19+-0.06}, with the remaining 90\% of AOD coming from \ce{SO2} emissions. Aerosol emission in South Asia is more diverse, with the increase in AOD from OC (\num[]{0.02+-0.08}; 9.1\%), BC (\num[]{0.03+-0.09}; 13.6\%), \ce{SO2} (\num[]{0.18+-0.12}; 82\%) of the total AOD of \num[]{0.22+-0.11}. That is, for present-day (2014) aerosol precursor emissions, \ce{SO2} is the primary contributor to AOD for all four selected regions.

Present-day atmospheric oxidants play an essential role in increasing AOD estimation. Aside from slightly lower \ce{SO2} emission in \histsst{} (averaged emission from 2005--2015) versus in \textit{piClim-aer} (2015), these two pairs of simulations are similar. The difference lies in the background state of the atmosphere. In the timeslice simulation, the control simulation \textit{piClim-control} features pre-industrial surface temperatures and short-lived climate forcing, including \ce{O3} and \ce{CH4}. The transient simulation pair, on the other hand, features surface temperatures and other emissions that are time-dependent; these features increase background oxidants for sulfate aerosol formation. This results in a 33\% increase in AOD in East and South Asia, emphasising the role of oxidants in aerosol formation.

% The UKESM1 AOD trends and global distribution were evaluated with satellite and ground-based observations in \citet{mulcahyDescriptionEvaluationAerosol2020}. UKESM1 consistently underestimated AOD at 550 nm relative to multiple satellite products. The model can simulate the global distribution of aerosols, showing AOD hotspots over East Asia, South Asia, and sub-Saharan Africa, consistent with observations.  


\begin{figure}[ht!]
    \centering
    \includegraphics[width=\linewidth]{Chapter5/Figs/regional_n50_series_histsst_sstpiaer.png}
    \caption{a) Global and regional time series of changes in tropospheric sulfate aerosol burden due to aerosol precursor emissions (\histsst{} minus \sstpiaer{}) from surface to 10 km altitude. b-c) Global distribution of changes in total sulfate aerosol burden due to aerosol precursor emissions between 1960--1990 and 2000--2015.}
    \label{fig:ch5:N50_series}
\end{figure}

In addition to AOD, \ce{SO2} oxidation also affects aerosol number concentration, which is sensitive to both the production pathway and the overall production rate. In UKESM1, \ce{SO2} oxidised in the gas phase produces sulfuric acid vapours which readily nucleate into new sulfate aerosol particles. However, \ce{SO2} that undergoes aqueous-phase oxidation is assumed to react in existing aerosol particles, and does not create new particles. 

Figure \ref{fig:ch5:N50_series} shows the time evolution of the number concentration of aerosols with a diameter above 50 nm due to increased aerosol precursor emissions. N50 concentration is shown here, as aerosols with a diameter above 50 nm act as cloud condensation nuclei. Globally, N50 increases by \qty{95.9}{\per\cubic\centi\metre}, while within the European boundary, the average aerosol number concentration more than doubled that of the global average to above \qty{240}{\per\cubic\centi\metre} between 1960 and 1990. Despite a 50\% higher total oxidation tendency and a sulfate aerosol mass burden, the N50 concentration in Europe is in the same ballpark as East North America. This is likely due to Europe's higher aqueous-phase oxidation, which does not form new aerosol particles. 

A similar pattern in particle density is also observed when comparing East and South Asia. The average increase in N50 number concentration in both regions is the same in 2015, even though the total oxidation in East Asia is double that in South Asia in the 2010s (Figure \ref{fig:ch5:tot_oxidation_series}). South Asia has high levels of gas-phase oxidation, which promotes sulfate aerosol particle nucleation, potentially explaining the higher new particle density per unit of oxidation tendency. Another explanation could be that aerosol particle coagulation imposes an upper bound on aerosol number concentration by increasing collisions and decreasing the number of particles at higher aerosol number density. As with sulfate burden and AOD, while much of the increase in aerosol number concentration is localised near the source, the change in aerosol number is seen beyond the defined region boundaries due to transport. 

So far, this section has shown that unique regional factors, such as the proportion of aqueous-phase oxidation driven by cloud cover, influence aerosol properties, especially the number concentration of cloud condensation nuclei per unit of oxidation.  



\subsection{Regional difference in historical cloud properties }

Aerosols play a role in cloud formation by serving as a germ for cloud condensation. Sulfate oxidation may provide cloud condensation nuclei, depending on whether \ce{SO2} is oxidised with OH or in existing cloud droplets with \ce{O3} and \ce{H2O2}. As shown in the previous section, the oxidation location plays an essential role in determining the amount of new aerosol particle formation. This section analyses changes in cloud properties resulting from aerosol precursor emissions across the four regions of interest during the historical period.


\begin{figure}[ht!]
    \centering
    \includegraphics[width=\linewidth]{Chapter5/Figs/regional_cdnc_series_histsst_sstpiaer.png}
    \caption{a) Global and regional time series of changes in cloud droplet number concentration (CDNC) due to aerosol precursor emissions (\histsst{} minus \sstpiaer{}) from surface to 10 km altitude. b-c) Global distribution of changes in CDNC due to aerosol precursor emissions between 1960--1990 and 2000--2015.}
    \label{fig:ch5:cdnc_series}
\end{figure}

With increased aerosol number, cloud droplet number concentration would increase as there are more nuclei for cloud droplets to condense onto. Figure \ref{fig:ch5:cdnc_series} shows the historical changes in average cloud droplet number concentration (CDNC) below 10 km due to increases in aerosol precursor emissions. Globally, aerosol precursor adds \qty{32.8}{\per\cubic\centi\metre} of CDNC, which maximises at 1975 and stabilises until 2015, as also discussed in Chapter \ref{ch3:title}. 

Regionally, in Europe, CDNC increases and plateaus in 1910 at \qty{52.4}{\per\cubic\centi\metre}, then exhibits another increase during the pothole period between 1960--1990 to \qty{94.1}{\per\cubic\centi\metre}. Thereafter, CDNC over Europe decreases rapidly to just slightly the global mean of \qty{40.1}{\per\cubic\centi\metre} in the present day.

CDNC over East North America shows the same rate of increase as Europe in the initial simulation years, from 1850 to 1910, but stabilises at a 30\% higher concentration, \qty{96.3}{\per\cubic\centi\metre}, than CDNC in Europe during the same period. Despite lower total oxidation in East North America than in Europe, CDNC over East North America shows a larger increase, consistent with the assumption that a greater fraction of gas-phase oxidation leads to more CCN formation and thus greater changes in cloud properties. 

An increase in aerosol precursor emissions over East Asia leads to a rapid rise in CDNC from 1950 onward, only to slow slightly after 1990 and reach a maximum in the present day at \qty{143.4}{\per\cubic\centi\metre}, 25\% higher than the highest level in East North America. This is expected, as the N50 concentration over East Asia is almost twice that over East North America during the same period.

In South Asia, the response of CDNC to the rise in aerosol precursor emissions resembles that of East Asia, with the same rate of change from 1950. Unlike East Asia, the rate of increase of CDNC does not dampen in 1990, leading to a substantial difference in the present day of \qty{194}{\per\cubic\centi\metre}.

The impact of aerosol precursor emissions extends far beyond their source regions, especially during the peak in European and Eastern North American emissions. Figure \ref{fig:ch5:cdnc_series}b shows that the increase in CDNC covers the whole of the Eurasian continent and the entirety of North America.
The impact of East and South Asian emissions is concentrated in the regions where they are emitted. CDNC changes are predominantly observed over continental areas. 

\begin{figure}[th!]
    \centering
    \includegraphics[width=\linewidth]{Chapter5/Figs/regional_Reff_series_histsst_sstpiaer.png}
    \caption{a) Global and regional time series of changes in effective cloud droplet radius (\Reff{}) due to aerosol precursor emissions (\histsst{} minus \sstpiaer{}) from surface to 10 km altitude. b-c) Global distribution of changes in \Reff{} due to aerosol precursor emissions between 1960--1990 and 2000--2015.}
    \label{fig:ch5:reff_series}
\end{figure}

% \begin{figure}[ht!]
%     \centering
%     \includegraphics[width=\linewidth]{Chapter5/Figs/regional_cl_seasons_histsst_sstpiaer.png}
%     \caption{Caption}
%     \label{fig:placeholder}
% \end{figure}

As the cloud droplet number increases with aerosol number, each cloud droplet shrinks in size, and the cloud droplet effective radius (\Reff{}) reflects these changes. Figure \ref{fig:ch5:reff_series} shows the impacts of the historical increase in aerosol precursor emissions on \Reff{} averaged in four regions below 10 km. Global mean \Reff{} decreases globally and more so regionally. 

\begin{figure}[th!]
    \centering
    \includegraphics[width=\linewidth]{Chapter5/Figs/regional_cod_series_histsst_sstpiaer.png}
    \caption{a) Global and regional time series of changes in cloud optical depth due to aerosol precursor emissions (\histsst{} minus \sstpiaer{}). b-c) Global distribution of changes in cloud optical depth due to aerosol precursor emissions between 1960--1990 and 2000--2015.}
    \label{fig:ch5:cod_series}
\end{figure}

In addition to cloud microphysical properties, aerosols affect cloud optical properties, as shown by changes in cloud optical depth (Figure \ref{fig:ch5:cod_series}). Cloud optical depths increase globally, with larger changes near sources of aerosol precursors.


This section shows that historical changes in aerosol precursor emissions alter cloud microphysical and optical properties with regional signatures. Europe shows lower CDNC sensitivity to total oxidation because a lower proportion of \ce{SO2} oxidation occurs in the gas phase. Meanwhile, South Asia shows greater changes in cloud properties due to higher gas-phase oxidation of \ce{SO2}. The susceptibility of CDNC to CCN density is one of the major factors controlling the highly uncertain change in the amount of solar radiation reflected by clouds, which is quantified by cloud optical depth, when aerosol emissions are perturbed. 


\subsection{Regional difference in historical ERF due to aerosols}

Radiative forcing by aerosols is the most important anthropogenic cooling component, and these interactions remain the largest source of uncertainty in radiative forcing estimates \citep{forsterEarthEnergyBudget2021, bellouinBoundingGlobalAerosol2020}. As with the preceding Chapters, aerosol effective radiative forcing (ERF) is decomposed into cloud radiative effects (\dcre{}), aerosol direct effect (\irf), and the albedo plus clean-and-clear-sky component (\ERFcsclean).


\begin{figure}[ht!]
    \centering
    \includegraphics[width=\linewidth]{Chapter5/Figs/regional_net_toa_series_histsst_sstpiaer.png}
    \caption{a) Global and regional time series of changes in effective radiative forcing (ERF) due to aerosol precursor emissions (\histsst{} minus \sstpiaer{}). b-c) Global distribution of changes in ERF due to aerosol precursor emissions between 1960--1990 and 2000--2015.}
    \label{fig:ch5:erf_series}
\end{figure}

Aerosols' effects on radiative imbalance at the top of the atmosphere are shown in Figure \ref{fig:ch5:erf_series} as a 10-year rolling annual mean ERF. Globally, aerosol precursors induce a negative ERF, meaning that more radiation leaves the atmosphere than enters it. Negative ERF induces this global deficit in many regions of the Earth. The global distribution of changes in ERF, Figure \ref{fig:ch5:erf_series}b-c), illustrates that the impact of aerosol precursor emissions reaches far beyond the emission sources. That is, ERF computed by averaging within the boundary underestimates the actual impact of emissions from that boundary. Nevertheless, ERF changes are most substantial near the source, so trend and correlation analyses are informative.

Over the defined Europe region, ERF is the most negative between 1960 and 1990 at \qty{-4.39}{W~m^{-2}} with the aerosol impact reaching beyond the drawn boundary into the northeastern region of the Eurasia continent, as shown in Figure \ref{fig:ch5:erf_series}. As aerosol precursor emissions cede in the area, the ERF becomes less negative, at \qty{-2.0}{W~m^{-2}} in 2000--2015.

Just like Europe, ERF over East North America is the most negative between 1960 and 1990 due to elevated aerosol precursor emissions, with the coverage of negative aerosol forcing extending far from the source regions into the North Atlantic Ocean and possibly Europe. However, as Europe's emissions are elevated during the same period, it is not possible to disentangle the overlap without simulations that target emissions from each region. 


The average ERF over the East Asia region increases over the historical period, reaching \qty{-3.95}{W~m^{-2}} between 2000 and 2015. However, the ERF in the South Asia region is relatively small, even though this region has been shown to exhibit a substantial decrease in \Reff{}. That is, aerosol ERF over this region is not sensitive to changes in cloud properties. However, the lack of ERF response over South Asia may be due to methodological limitations. Just beyond the region boundary, Figure \ref{fig:ch5:erf_series}c) shows a strong negative aerosol ERF in the Indian Ocean region, which may imply that aerosol from South Asia impacts the ERF but not directly at the emission sources. 


\begin{figure}[ht!]
    \centering
    \includegraphics[width=\linewidth]{Chapter5/Figs/regional_erf_series_histsst_sstpiaer_diff.png}
    \caption{Regional 10-year rolling mean time series of effective radiative forcing and its components due to aerosol precursor emissions (\histsst{} minus \sstpiaer{}) over a) Europe, b) East North America, c) East Asia, and d) South Asia.}
    \label{fig:ch5:erf_series_by_region}
\end{figure}

Aerosol ERF in each region is decomposed into cloud radiative effect (\dcre{}), aerosol direct effects (\irf{}) and a combination of surface albedo plus clear-sky-aerosol-free (\ERFcsclean{}) in Figure \ref{fig:ch5:erf_series_by_region}. In Europe, the primary source of ERF is from aerosol-cloud interaction. 

\change[inline]{more discussion here if Fig \ref{fig:ch5:erf_series_by_region} is still needed after writing next section}

\subsection{The sources of sensitivity in aerosol ERF}

This chapter has explored the historical trends in the production of sulfate aerosols and their effects on the radiative balance, and has shown that each step leading to aerosol radiative forcing can be linked to prior steps, with substantial variation across regions. This section applies the theoretical framework to quantify the sensitivity of each process from emission that leads to radiative effects to inform the role of each process and the importance of emission location on the eventual ERF results.

The sensitivity of a response to a driver is obtained from the slope of a linear relationship between the driver and the response, fitted using a root-mean-square error (RMSE) regression model. Each data point is the difference between the annual mean of the difference between \histsst{} and \sstpiaer{} in both the driver and the response over a defined region. Table 
\ref{ch5:tab:sensitivity}, summarises each of the sensitivity terms as defined in Relationship \ref{ch5:eq:flow-of-response}.


\clearpage
\begin{landscape}
\begin{table}[htbp]
\centering
\small
\setlength{\tabcolsep}{5pt}
\caption{Sensitivity of a response to a driver due to a change in historical precursor emission. }
\label{ch5:tab:sensitivity}
\begin{tabular}{lllrrrrrrrrrr}
\hline
\multirow{2}{*}{$\gamma$} & \multirow{2}{*}{driver} & \multirow{2}{*}{response} & \multicolumn{2}{c}{Global} & \multicolumn{2}{c}{Europe} & \multicolumn{2}{c}{East N. America} & \multicolumn{2}{c}{East Asia} & \multicolumn{2}{c}{South Asia} \\
\cmidrule(lr){4-5} \cmidrule(lr){6-7} \cmidrule(lr){8-9} \cmidrule(lr){10-11} \cmidrule(lr){12-13}
& & & Slope & $R^2$ & Slope & $R^2$ & Slope & $R^2$ & Slope & $R^2$ & Slope & $R^2$ \\
\hline
1 & $P_{\ce{SO2}}$ & $Q_{\ce{SO2}}$ & \num{5.15e-03} & \num{1.00} & \num{3.22e-03} & \num{0.99} & \num{3.19e-03} & \num{1.00} & \num{3.66e-03} & \num{1.00} & \num{4.62e-03} & \num{1.00} \\
2 & $Q_{\ce{SO2}}$ & $P_{\text{S(VI)}}$ & \num{7.87e+01} & \num{0.99} & \num{7.38e+01} & \num{0.99} & \num{9.80e+01} & \num{0.99} & \num{9.24e+01} & \num{0.99} & \num{9.83e+01} & \num{1.00} \\
- 2i & $Q_{\ce{SO2}}$ & $P_{\ce{SO2 + OH}}$ & \num{3.77e+01} & \num{0.94} & \num{2.53e+01} & & \num{4.40e+01} & & \num{4.07e+01} & & \num{7.72e+01} & \\
- 2ii & $Q_{\ce{SO2}}$ & $P_{\ce{SO2 + O3}}$ & \num{1.70e+01} & \num{0.94} & \num{3.83e+01} & & \num{2.86e+01} & & \num{2.33e+01} & & \num{1.99e+00} & \\
- 2iii & $Q_{\ce{SO2}}$ & $P_{\ce{SO2 + H2O2}}$ & \num{2.40e+01} & \num{0.98} & \num{1.02e+01} & & \num{2.55e+01} & & \num{2.83e+01} & & \num{1.92e+01} & \\
3 & $P_{\text{S(VI)}}$ & $Q_{\text{S(VI)}}$ & \num{1.59e-02} & \num{0.99} & \num{5.82e-03} & \num{0.95} & \num{5.67e-03} & \num{0.96} & \num{6.47e-03} & \num{0.99} & \num{1.06e-02} & \num{0.90} \\
4a & $Q_{\text{S(VI)}}$/area & $\tau_a$ & \num{4.69e+07} & \num{0.97} & \num{6.23e+07} & \num{0.98} & \num{5.64e+07} & \num{0.98} & \num{6.72e+07} & \num{0.99} & \num{5.09e+07} & \num{0.96} \\
5a & $\tau_a$ & ERF & \num{-3.60e+01} & \num{0.78} & \num{-2.48e+1} & \num{0.46} & \num{-3.78e+01} & \num{0.70} & \num{-1.42e+01} & \num{0.38} & \num{-6.89e+00} & \num{0.20} \\
4b & $Q_{\text{S(VI)}}$/area & N50 & \num{1.16e+11} & \num{0.99} & \num{8.68e+10} & \num{0.95} & \num{1.04e+11} & \num{0.97} & \num{1.13e+11} & \num{0.99} & \num{9.76e+10} & \num{0.95} \\
5b & N50 & $N_d$ & \num{3.47e-01} & \num{0.96} & \num{3.65e-01} & \num{0.96} & \num{4.07e-01} & \num{0.92} & \num{3.57e-01} & \num{0.91} & \num{4.39e-01} & \num{0.98} \\
6b & $N_d$ & $\tau_c$ & \num{2.44e-04} & \num{0.88} & \num{2.22e-04} & \num{0.42} & \num{2.12e-04} & \num{0.67} & \num{1.44e-04} & \num{0.65} & \num{7.60e-05} & \num{0.15} \\
7b & $\tau_c$ & ERF & \num{-3.09e+00} & \num{0.86} & \num{-1.40e+00} & \num{0.17} & \num{-2.26e+00} & \num{0.61} & \num{-1.77e+00} & \num{0.56} & \num{-5.87e-01} & \num{0.05} \\
\hline
\end{tabular}
\end{table}
\end{landscape}



\begin{figure}[ht!]
    \centering
    \includegraphics[]{Chapter5/Figs/sensitivity_demiso2_dso2_histsst_sstpiaer.png}
    \caption{Domain-mean response of \ce{SO2} burden ($\Delta Q_{\ce{SO2}}$) to historical \ce{SO2} emission ($\Delta P_{\ce{SO2}}$) changes. Each data point is the annual mean of the driver from the perturbed (\histsst{}) subtracted by the conntrol (\sstpiaer{}) simulation, plotted against the response for the four region domains and the global domain.}
    \label{fig:ch5:emiso2_vs_so2}
\end{figure}

The first step in the cascade of responses to \ce{SO2} emissions is a change in \ce{SO2} burden. Figure \ref{fig:ch5:emiso2_vs_so2} shows that global and regional \ce{SO2} burden responds linearly to emission driver with the coefficient of determination ($R^2$) of 0.99--1.00 in all regions. All regions show lower sensitivity than the global relationship, and amongst all regions, South Asia has the highest sensitivity of \ce{SO2} burden to emissions. The loss of \ce{SO2} determines burden and its response to emissions, with greater loss corresponding to lower sensitivity of burden to emissions. That is, total \ce{SO2} removal is efficient in Europe and East North America, but less so in South Asia and globally.

\begin{figure}[ht!]
    \centering
    \includegraphics[width=\linewidth]{Chapter5/Figs/sensitivity_doxi_demiss_histsst_sstpiaer.png}
    \caption{Response of secondary sulfate production ($\Delta P_{\text{S(VI)}}^i$) to \ce{SO2} burden ($\Delta Q_{\ce{SO2}}$) changes. Each data point is the annual mean of the driver from the perturbed (\histsst{}) simulation, subtracted from the control (\sstpiaer{}) simulation, plotted against the response for the four regional domains and the global domain. \change[inline]{The y-axis label needs to be changed to oxidation or just remove the label, leaving only the unit.}}
    \label{fig:ch5:so2_vs_oxidation}
\end{figure}

\ce{SO2} oxidation is one of the processes which remove \ce{SO2} from the atmosphere and is the next immediate step in the process of aerosol formation. Figure \ref{fig:ch5:so2_vs_oxidation} shows the sensitivity of each of the oxidation pathways to an increase in \ce{SO2} burden, driven by the historical aerosol precursor emissions. Here, the region plays a significant role in determining the sensitivity. Europe exhibits the strongest sensitivity of \ce{SO2 + O3} to the increase in burden, while \ce{SO2 + H2O2}, another aqueous-phase oxidation route, is only 25\% as sensitive. Apart from Europe, \ce{SO2 + OH} oxidation is the most sensitive of the three oxidation pathways to an increase in \ce{SO2} burden. Gas-phase oxidation over South Asia is the most sensitive to changes in burden, showing the sensitivity of \qty{77.2}{yr^{-1}}, which is double that of the global sensitivity. 


\begin{figure}[ht!]
    \centering
    \includegraphics{Chapter5/Figs/sensitivity_dtotoxi_demiss_histsst_sstpiaer.png}
    \caption{a) Regional response of total secondary sulfate aerosol production ($\Delta P_{\text{S(VI)}}$) to historical \ce{SO2} burden ($\Delta Q_{\ce{SO2}}$) changes. b) Global response to changes in \ce{SO2} burden ($\Delta Q_{\ce{SO2}}$) on secondary sulfate aerosol production from each oxidation pathway and total oxidation. Each data point is the annual mean of the driver from the perturbed (\histsst{}) subtracted by the control (\sstpiaer{}) simulation, plotted against the response for the four region domains and the global domain.\change[inline]{yaxis unit label is faulty :/ and the grey shading does not make sense.}}
    \label{fig:ch5:so2_vs_totoxi}
\end{figure}

The total secondary production is the sum of production from all three pathways, and its sensitivity to changes in burden indicates the efficiency of sulfate aerosol production within a defined domain. Figure \ref{fig:ch5:so2_vs_totoxi}a) shows the regional and global sensitivity, $\gamma_2$, which is a linear response with $R^2$ greater than 0.99, demonstrating a strong correlation in all regions. Global $\gamma_2$ is \qty{78.7}{yr^{-1}} and all regions except Europe have higher sensitivity than the global value. This shows that higher emissions do not imply equal growth in oxidation. Local environmental conditions control deposition and oxidation, thereby altering sensitivity.


\begin{figure}[ht!]
    \centering
    \includegraphics{Chapter5/Figs/sensitivity_dtot_so2_oxi_dmmrso4_histsst_sstpiaer_zoom.png}
    \caption{Response of sulfate aerosol burden ($\Delta Q_{\text{S(VI)}}$) to secondary sulfate aerosol production ($\Delta P_{\text{S(VI)}}$) changes. Each data point is the annual mean of the driver from the perturbed (\histsst{}) simulation, subtracted from the control (\sstpiaer{}) simulation, plotted against the response for the four regional domains and the global domain.}
    \label{fig:ch5:so2_oxi_vs_so4}
\end{figure}

Sulfate aerosol production contributes to the total sulfate burden, except for the sulfate lost through wet and dry deposition. Figure \ref{fig:ch5:so2_oxi_vs_so4} shows the response of sulfate aerosol burden to the change in secondary sulfate aerosol production. Globally, sulfate aerosol burden responds well to total oxidation, with greater sensitivity than in any region. 

\begin{figure}[ht!]
    \centering
    \includegraphics{Chapter5/Figs/sensitivity_dso4_per_area_dod550aer_histsst_sstpiaer.png}
    \caption{Response of aerosol optical depth ($\Delta \tau_a$) to sulfate aerosol burden per area ($\Delta Q_{\text{S(VI)}}/area$) changes. Each data point is the annual mean of the driver from the perturbed (\histsst{}) subtracted by the control (\sstpiaer{}) simulation, plotted against the response for the four region domains and the global domain.}
    \label{fig:ch5:so4_pa_vs_aod}
\end{figure}

% CCN vs CDNC
In relation to the existing literature, \citet{virtanenHighSensitivityCloud2025} quantified the sensitivity---referred to as susceptibility, S---of cloud droplet number concentration to cloud condensation nuclei (CCN) concentration by comparing five Earth system models, including UKESM1, with in-situ and satellite observations. Compared to measurements, UKESM1 showed lower CDNC sensitivity to CCN in the semi-urban area but greater sensitivity in the remote Arctic. Amongst the ESMs included in this study, UKESM1 showed greater sensitivity than three models, placing it amongst the more sensitive models.



\section{Conclusions}

I could have analysed the change in N50 due to each component of aerosol precursors. Still, from the estimated contribution to AOD, it is reasonable to assume that SO2 is the primary contributor to N50. But let us consider this as future work. And it just does not add to the argument. Just a bit of confirmation that change n50 is indeed due to SO2 and rule out bc and oc. 


Relate to other models

New framework for understanding ERF uncertainty across models

% -------

% \subsection{the change in sensitivity due to changes in oxidants}
% (compare with sstpio3, sstpich4)

% \begin{figure}[ht!]
%     \centering
%     \includegraphics[width=\linewidth]{Chapter5/Figs/regional_oxidation_seasonal_histsst_sstpich4.png}
%     \caption{Caption}
%     \label{fig:placeholder}
% \end{figure}

% ch4 impacts aerosol formation globally, which is the nature of ch4 long lifetime.

% \begin{figure}[ht!]
%     \centering
%     \includegraphics[width=\linewidth]{Chapter5/Figs/regional_oxidation_seasonal_histsst_sstpio3.png}
%     \caption{Caption}
%     \label{fig:placeholder}
% \end{figure}


% o3 precursors have shorter lifetimes so their impacts are localised.  Their total effect to increase aeorsol production especially in Europe.

% It is now possible to make more generalised conclusion about the impact of oxidants with sensitivity analysis. This may not need any more plots in the chapter (but may add plots in supplementary) plots but need a larger summary table

% [To add: summary table of sensitivities due to change in O3 precursors and CH4 over different regions]

% \newpage

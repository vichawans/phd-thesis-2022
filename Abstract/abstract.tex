% ************************** Thesis Abstract *****************************
% Use `abstract' as an option in the document class to print only the titlepage and the abstract.
\begin{abstract}


Aerosols moderate Earth’s atmospheric radiation budget through scattering and absorption of solar radiation, serve as the nucleating particles for cloud formation, and their properties modulate cloud properties such as brightness, coverage and lifetime. To assess the effects of aerosol on climate, global earth system models are an important tool. These model's capability to simulate detailed atmospheric chemistry coupled with aerosol process opens up a new avenue for research and questions regarding chemistry and aerosol interactions and their roles in the Earth’s climate system. 

Using the UK Earth System Model 1 (UKESM1), this thesis investigates sulfate aerosol formation over the CMIP6 historical period, 1850 to 2014. The output from the UKESM1 for the Aerosol Chemistry Model Intercomparison Project (AerChemMIP) experiments is used to show that methane and ozone precursors, which contribute to global warming, also play crucial roles in modifying aerosol formation. By analysing attribution experiments, this thesis shows that methane suppresses the formation of cloud condensation nuclei, further warming the climate beyond previous estimations. Ozone precursors, on the other hand, contribute to aerosol formation, cooling down the climate.  The seasonal cycle in radiative forcing is analysed, and the fate of sulfur dioxide emissions is discussed in terms of the branching between deposition and oxidation pathways. The radiative forcing of sulfur dioxide emitted in summer is stronger, i.e., it is more cooling, because the balance between deposition and oxidation favours oxidation in summer, forming more aerosol particles. Finally, it shows that emission location determines the branching between deposition and oxidation, with a higher fraction of sulfur dioxide being oxidised into sulfate aerosols in Northeast America and South Asia. 

Ultimately, this work improves our process-level understanding of Earth system models that interactively simulate aerosol from precursors and aims to improve the accuracy of aerosol radiative forcing predictions.



\end{abstract}

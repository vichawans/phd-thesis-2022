\chapter{Regional aerosol formation in the CMIP6 historical period}
% **************************** Define Graphics Path **************************
\ifpdf
    \graphicspath{{Chapter5/Figs/Raster/}{Chapter3/Figs/PDF/}{Chapter5/Figs/}}
\else
    \graphicspath{{Chapter5/Figs/Vector/}{Chapter5/Figs/}}
\fi


\section*{Abstracts}
This section focuses on regional aspects of aerosol formation. It deals with both annual and seasonal trends. The region of interest includes Northeastern America, Europe, Eastern Asia and South Asia.

NEA and EUR share similar emission trends where SO2 emissions peaked around 1980 and declined. Whereas, emissions from EAS and SAS are on the increase in the near present. 

\section{Introduction}
A modelling study shows that the production of oxidant over different region may be different \cite{zhangTroposphericOzoneChange2016}. Another model study evaluate how black carbon aerosol emission at diffent location affect ERF differently 

\subsection{SO2 emissions trends over different regions}

Explain the emissions trends over NEA, EUR, EAS, SAS. 

\subsection{Oxidant level}

\subsection{Annual cycle of oxidant trends and their drivers for different regions}



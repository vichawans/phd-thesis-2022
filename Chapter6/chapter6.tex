\chapter{Summary, Conclusions and Future Directions}
% **************************** Define Graphics Path **************************
\ifpdf
    \graphicspath{{Chapter6/Figs/Raster/}{Chapter6/Figs/PDF/}{Chapter6/Figs/}}
\else
    \graphicspath{{Chapte6/Figs/Vector/}{Chapter6/Figs/}}
\fi


\section{Summary of key results}

The main aim of this thesis has been to ...

\begin{itemize}
    \item [Chapter 3] aims to understand sulfate aerosol formation drivers in UKESM1 over the historical period. This chapter explores the global trends of sulfur budget and oxidation tendencies. Over the historical period, emissions of SLCFs, such as \ce{CH4} and \ce{O3} precursors, have changed, impacting the atmospheric trace gases.

    Oxidant level changes have a downstream effect on sulfate aerosol production since \ce{SO2} oxidation produces the majority of sulfate aerosols in the atmosphere. This cascade of interactions may impact aerosol radiative effects. 

    Given this, Chapter 3 investigates the following questions: 

    \begin{enumerate}
        \item  How do oxidant levels respond to changes in SLCF emissions over the historical period?
        \item  How do \ce{SO2} budget and oxidation tendencies respond to changes in oxidant levels over the historical period?
        \item How does sulfate aerosol size distribution respond to oxidant perturbation?
        \item At the process level, how do oxidant changes affect aerosol radiative forcing due to changes in oxidants?
    \end{enumerate}

    \item[Chapter 4] focuses on the seasonal variability of sulfate aerosol formation. Sulfate aerosol formation occurs on a sub-annual basis, with sulfate aerosol lifetime in the order of 7--14 days. This short lifetime poses a question on the impact of seasonal variability of oxidants and cloud droplets on aerosol formation. 

    UKESM1 has the capacity to simulate both aerosol mass and number independently, making it one of the most robust climate models regarding aerosol simulation. This makes UKESM1 a valuable tool for studying the aerosol properties in different seasons.
    s
    As mentioned in the previous section, one of the challenges for ESMs is anomalous cooling between 1960 and 1990, which was attributed to aerosol radiative effects \citep{zhangRoleAnthropogenicAerosols2021}. 

    Given this, Chapter 4 investigates the following questions: 

    \begin{enumerate}[resume]
        \item How do changes in seasonal levels of oxidants affect aerosol formation and subsequent radiative effects?
        \item How does UKESM1 perform regarding simulating GSAT seasonal variability?
    \end{enumerate}

    \item[Chapter 5] explores regional sulfate formation and regional differences due to the chemical background. In the historical period, anthropogenic emissions, including \ce{SO2}, increased with industrial activities in the European region and peaked in 1980 before decreasing. Then, emissions increased sharply in Eastern Asia and peaked in the 2000s. Since \ce{SO2} emission location dictates the background of available precursors and the atmospheric chemistry condition near the equator differs from that of the extratropical region, it is possible that different regions have different baseline oxidation tendencies and also behave differently to oxidant level changes. 
    
    This Chapter further inspects the details in regions, aiming to discover oxidation characteristics in different periods and regions.  
        
    Given this, Chapter 5 investigates the following questions: 
    
   \begin{enumerate}[resume]
    \item What is the effect of emission location change on oxidation tendencies?
    \end{enumerate}
\end{itemize}


\section{Future directions}



Despite the known effect of aerosol on atmospheric chemistry \citep[e.g.,][]{oconnorApportionmentPreIndustrial2022}, sulfate aerosol modelling is still an active area of study with many open questions. For example, as a large number of chemical species slows down the progress of the study due to computational resource limits, there remain gaps in treatment of aerosol composition (e.g., nitrate, secondary organic aerosols) in many regions \citep[e.g.,][]{mulcahyDescriptionEvaluationAerosol2020}, and parameterizations of sub-grid processes add to the uncertainty of model prediction so that much of the feedback between aerosol and other components of the Earth system is not yet fully explored \citep{seikiImprovementGlobalCloudSystemResolving2015}. 

Aerosol contribute to large uncertainty in future warming \citep{watson-parrisLargeUncertaintyFuture2022,wilcoxRegionalAerosolModel2022}

\subsection{piSO2 simulations}

\subsection{Region-focused simulations for SO2 and aerosol impacts on climate for historical emissions}

\subsection{Expanding the analysis to other ESMs}

\subsection{Looking beyond \texorpdfstring{\ce{SO2}}{SO2} oxidation}
how about black carbon and organic soa? Does this affect ERF only seasonally? \citep{bellouinRegionalSeasonalRadiative2016}



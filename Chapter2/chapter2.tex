%!TEX root = ../thesis.tex
%*******************************************************************************
%****************************** Second Chapter *********************************
%*******************************************************************************

\chapter{Data and Methods}

\ifpdf
    \graphicspath{{Chapter2/Figs/Raster/}{Chapter2/Figs/PDF/}{Chapter2/Figs/}}
\else
    \graphicspath{{Chapter2/Figs/Vector/}{Chapter2/Figs/}}
\fi

\section{CMIP6 and AerChemMIP experiment design}

This research focuses on \ce{SO2} oxidation and aerosol formation and utilizes data from the AerChemMIP transient historical atmosphere-only prescribed SST simulations \citep{collinsAerChemMIPQuantifyingEffects2017}. These simulations were designed to calculate transient ERFs that drive climate change.  While the simulations were not precisely designed for the purpose of this research, their configurations and emissions align well this research purpose.


Table \ref{tab:2.exps} shows the simulation configurations with all historical transient experiments from AerChemMIP that experiments cover the period between 1850 and 2014. "AOGCM" means atmosphere-ocean coupled simulation. "AGCM" means atmosphere-only simulation. The "AER" suffix means the models should at least calculate tropospheric aerosol driven by emission fluxes. The "CHEM$^\text{S}$" suffix means at least stratospheric chemistry is required; "CHEM$^\text{T}$" means at least tropospheric chemistry is required. "Hist" means the concentration or emission should evolve as for the CMIP historical simulation. "1850" means the concentrations or emissions should be fixed to the year 1850. 

\begin{table}
   \caption[AerChemMIP experiments related to this work]{A summary of AerChemMIP experiments related to this work.}
   \label{tab:2.exps}
   \centering
   \begin{tabular}{l p{30mm} l p{18mm} p{18mm} l}
    \toprule
     Experiment ID & Minimum model configuration & \ce{CH4} & Aerosol precursors & \ce{O3} precursors & suite-id \\
    \midrule
     \textit{historical}      & AOGCM AER & Hist & Hist & Hist & u-bc179\\
     \textit{hist-piAer}      & AOGCM AER & Hist & 1850 & Hist & u-bg705\\
     \textit{hist-piNTCF}     & AOGCM AER & Hist & 1850 & 1850 & u-bg946\\
     \textit{histSST}         & AGCM AER & Hist & Hist & Hist & u-bh626\\
     \textit{histSST-piAer}   & AGCM AER & Hist & 1850 & Hist & u-bi541\\
     \textit{histSST-piO3}    & AGCM CHEM$^{\text{T}}$ & Hist & Hist & 1850 & u-bl670\\
     \textit{histSST-piCH4}   & AGCM CHEM$^{\text{T/S}}$ & 1850 & Hist & Hist & u-bl551\\
     \textit{histSST-piNTCF}  & AGCM AER & Hist & 1850 & 1850 & u-bl277\\
     \bottomrule
   \end{tabular}
\end{table}

The CMIP6 requires that participating models archives a standardized set of variables that are processed in  specific ways. For example, the AerChemMIP archives sulfate aerosol formation tendency in two variables: \texttt{cheaqpso4} and \texttt{chegpso4}. These are the aqueous- and gas-phase production rate of \ce{SO4}, respectively. While this is useful for model intercomparison studies, \ce{O3} and \ce{H2O2} oxidation are added up to \texttt{cheaqpso4}, making it impossible to perform analysis based on individual channels. As such, the native model data is more useful for in-depth analysis specific to a model. This project uses both the processed AerChemMIP archive and UKESM1 native data.


\section{Definitions of terms and methods}
\label{sec:2.terms}
The main definitions used in this work are related to the concept of atmospheric chemical budget. The budget of a trace gas consists of three quantities: its global source, global sink and atmospheric burden. For the sulfur budget, all masses are represented in Tg of sulfur, not in Tg of the whole compound.

\subsubsection{Source / Emission}

The source of a trace gas (Tg yr$^{-1}$) is the sum of all sources including primary emission from surface emission and \textit{in situ} chemical production also called secondary emission. For sulfur dioxide, the source terms include primary emission from both natural and anthropogenic sources, and secondary emission from oxidation of \ce{H2S}, DMS and dimethyl disulfide. CMIP6 uses various sources of \ce{SO2} emission as listed in section \ref{sec:1.ukesm1}. In UKESM1, source terms are outputted in mol s$^{-1}$ per grid box for each type of emission and are referred to as fluxes. SO4 source terms comprise primary emission which is 2.5\% of anthropogenic \ce{SO2} emission and secondary emission from \ce{SO2} oxidation.

\subsubsection{Sink / Loss}

Similar to source terms, sinks of a trace gas (Tg yr$^{-1}$) is the sum of all sinks or losses, which could be wet and dry deposition, and \textit{in situ} chemical loss. UKESM1 outputs source terms in mol s$^{-1}$ per grid box for each type of emission. \ce{SO2} sink terms include wet and dry deposition and chemical loss via oxidation with OH, \ce{O3} and \ce{H2O2}. \ce{SO4} is loss via wet and dry deposition.

\subsubsection{Burden}

The burden (Tg) is defined as the total mass of the gas integrated over the atmosphere. The burden could be calculated for related reservoirs, for example, troposphere only or both troposphere and stratosphere. In this work, only tropospheric burden is considered for all species including \ce{SO2}, \ce{SO4}, OH, \ce{O3} and \ce{H2O2}.

\subsubsection{Lifetime/residence time}

The lifetime of a trace gas is defined as the average time that a molecule of that species remains in a reservoir before removal. Atmospheric lifetimes vary from seconds for reactive radicals to years for stable molecules. Lifetimes or residence time of a substance is different from the \textit{e-folding lifetime of a reaction} (also known as the \textit{mean lifetime of A against a reaction}). 

If the reservoir is the whole atmosphere and is in a steady state, that is the burden is considered constant,  the atmospheric lifetime of a compound is estimated from 

\begin{equation}
\label{eq:lifetime}
 \text{Lifetime} = \frac{\text{Burden}}{\text{Loss}}    
\end{equation}

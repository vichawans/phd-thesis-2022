%!TEX root = ../thesis.tex
%*******************************************************************************
%*********************************** First Chapter *****************************
%*******************************************************************************

\chapter{Introduction and motivation} 
\label{ch1:title}

This doctoral thesis includes two chapters that are being prepared for submission. The Introduction provides an overview of the research background, questions, and thesis structure. Each result chapter contains an introduction to the relevant research questions.

\section{Earth's energy balance and aerosols}

Interactions among aerosols, clouds, radiation, and atmospheric chemistry are central to determining atmospheric composition and Earth's energy balance \citep{forsterEarthEnergyBudget2021}. Understanding these interactions is essential for reconstructing past climate variations and predicting future climate change. 

Earth's climate results from a delicate balance between incoming solar energy and outgoing terrestrial radiation. Solar radiation is the primary source of Earth's energy. To retain the energetic balance, the Earth must radiate an equal amount of energy back into space. On average, approximately 29\% of incoming shortwave solar radiation is reflected by the atmosphere and Earth's surface, while 23\% is absorbed by the atmosphere and 47\% by the surface \citep{readGlobalEnergyBudgets2016}. The absorbed shortwave radiation heats the surface, which in turn emits longwave (infrared) radiation. This thermal radiation is partially absorbed by atmospheric constituents, including greenhouse gases, and re-emitted both upward to space and downward toward the surface, contributing to the greenhouse effect. This global energy exchange is illustrated in Figure \ref{fig:trenberth}, adapted from \citet{readGlobalEnergyBudgets2016}, summarising the radiative fluxes in Earth's climate system. In equilibrium, the absorbed solar energy is balanced by the energy emitted or reflected back to space.

\begin{figure}[H]
    \centering
    \includegraphics[width=0.7\columnwidth]{Chapter1/figs/01_trenberth.jpg}
    \caption[The Earth’s energy budget]{The Earth’s energy budget from \citet{readGlobalEnergyBudgets2016}, first used in \citet{kiehlEarthAnnualGlobal1997}. The blue arrows show the incoming shortwave solar radiation, and red arrows show the outgoing longwave radiation from the Earth's surface.}
    \label{fig:trenberth}
\end{figure}

Any natural or anthropogenic factor that changes the balance of Earth's energy budget is called a climate forcing agent \citep{ramaswamyRadiativeForcingClimate2018}. These agents can warm or cool the planet by causing net radiative gain or loss to the energy budget. For instance, greenhouse gases warm the planet by absorbing outgoing longwave radiation and trapping energy within the Earth \citep[e.g.,][]{ramaswamyRadiativeForcingClimate2018,readGlobalEnergyBudgets2016,kiehlEarthAnnualGlobal1997}. 

Once a climate forcing agent perturbs the Earth's energy out of balance, the climate system responds to restore equilibrium through a change in the temperature of the surface and atmosphere. The temperature change can induce other surface and atmospheric responses, which are termed feedbacks, such as the amount of atmospheric water vapour and cloud, further impacting the net energy loss \citep{ramaswamyRadiativeForcingClimate2018}. To aid quantification and comparison of climate forcing agents, the concept of radiative forcing, which is an analytical framework for defining and quantifying Earth's radiative energy balance perturbations, is widely adopted \citep[e.g.,][]{ipccClimateChange19901992,ipccClimateChange20132014,ramaswamyRadiativeForcingClimate2018,ramanathanGreenhouseEffectDue1975}. 

When the Earth system is perturbed by a forcing agent, the global surface air temperature (GSAT) responds to the radiative perturbation, which gives rise to an energy imbalance at the top of the atmosphere (TOA), which is termed effective radiative forcing \citep[ERF; ][]{forsterEarthEnergyBudget2021}. Commonly, ERF is defined with respect to the state of the climate before industrialisation, either with 1750 or 1850 \citep[e.g., ][]{forsterEarthEnergyBudget2021,taylorOverviewCMIP5Experiment2012, eyringOverviewCoupledModel2016}. The relationship between radiative flux and the response could be written as the linear energy budget equation \citep{forsterEarthEnergyBudget2021}.

\begin{equation}
\label{eq:erf-short}
    \Delta N = \Delta F - \alpha \Delta T
\end{equation}

Here, $\Delta N$ (W m$^{-2}$) denotes the change in the top-of-atmosphere net downward radiative flux, which is the result of the ERF caused by a climate forcing agent, $\Delta F$ (W m$^{-2}$). $\alpha$ is the net feedback parameter and $\Delta T$ ($^\circ$C) is the response in global surface air temperature. 

Figure \ref{fig:1.ERF} shows the pre-industrial to present-day effective radiative forcing estimates for 1750 to 2019 for the change in forcing agents as featured in the Intergovernmental Panel on Climate Change (IPCC) Assessment Report (AR) 6 \citep{forsterEarthEnergyBudget2021}. The IPCC Assessment Reports are among the most authoritative resources guiding global policy and scientific discourse on climate change. These reports synthesise observational data, model output, and process understanding to contextualise many aspects of the climate, including how chemical species interact with the climate system and affect radiative forcing, air quality, and environmental health. It could be seen that carbon dioxide (\ce{CO2}), methane (\ce{CH4}), and a wide range of greenhouse gases have positive radiative forcing, which means they exert a warming effect on global surface air temperature. Atmospheric aerosols, such as organic carbon, black carbon, ammonia and sulfate (formed from precursor gas sulfur dioxide (\ce{SO2})), have negative ERF and contribute to a decrease in GSAT.

\begin{figure}[H]
    \centering
    \includegraphics[width=\linewidth]{Chapter1/figs/SRL-image-11.png}
    \caption[Effective radiative forcing (ERF) from 1750 to 2019]{Contribution to (a) effective radiative forcing (ERF) and (b) global mean surface air temperature (GSAT) change from component emissions between 1750 and 2019. Figure from \citet{szopaShortlivedClimateForcers2021}, with data based on \citet{thornhillEffectiveRadiativeForcing2021}.}
    \label{fig:1.ERF}
\end{figure}


% \section{Quantifying radiative effects of aerosols using effective radiative forcing}

% Climate impact of aerosols
Atmospheric aerosol particles (hereafter referred to as "aerosols") are solid or liquid particles suspended in the air, including dust, pollutants, and natural substances such as sea salt and pollen. Aerosols are a climate forcing agent that modifies the Earth's energy budget, both directly via scattering or absorption, and indirectly via latent heat release, evapotranspiration and enhancing cloud formation and by modulating the cloud properties \citep[e.g., ][]{angstromAtmosphericTransmissionSun1929, readGlobalEnergyBudgets2016}. Aerosol-radiation interaction (ARI) or aerosol direct effect refers to the impact of aerosols on shortwave incoming radiation. Aerosols with high reflectivity and a low imaginary refractive index scatter solar radiation, while low-reflectivity aerosols absorb radiation and trap energy within the atmosphere \citep{angstromAtmosphericTransmissionSun1929, tuccellaModelingBlackBrown2025}. In addition, aerosols influence cloud micro- and macrophysical properties through the process collectively called the aerosol–cloud interaction (ACI). ACI arises because aerosols act as cloud condensation nuclei, allowing water vapour to condense at lower temperatures than homogeneous nucleation. Thus, the change in aerosol number and size modifies cloud properties \citep[e.g., ][]{rosenfeldGlobalObservationsAerosolcloudprecipitationclimate2014, boucherCloudsAerosols2014,persadAerosolsMustBe2022}. That is, an increase in aerosol in the cloud increases the cloud's albedo and lifetime, leading to weather and climate effects \citep{twomeyInfluencePollutionShortwave1977, albrechtAerosolsCloudMicrophysics1989}. Compared to the clean atmosphere, clouds in the regions with higher below-cloud aerosol concentrations exhibit smaller cloud droplets (reduced effective cloud radius (\Reff{}) and increased cloud droplet number concentration (CDNC) and enhanced cloud liquid water contents and optical depths, leading to longer lifetime and lower albedo for the clouds \citep[e.g., ][]{twomeyInfluencePollutionShortwave1977,twohyEvaluationAerosolIndirect2005}. 


Using the concept of ERF, the radiative effects due to aerosols can be separated into aerosol-radiation interaction (ERFari) and aerosol-cloud interaction \citep[ERFaci;][]{ghanTechnicalNoteEstimating2013}. As shown in Figure \ref{fig:1.ERF}, ACI is the cause of the largest anthropogenic climate forcing uncertainty. ERFaci is deemed the only radiative forcing with low confidence in the latest IPCC Assessment Report \citep{forsterEarthEnergyBudget2021}. Table \ref{tab:ipcc-rf-summary} shows the estimated aerosol radiative forcing reported by the IPCC. Despite successive efforts to assess and summarise aerosol radiative effects, they have only been slightly bounded. Between the AR5 and AR6, aerosol-radiation effects have been tightly bound, with the range (5--95\%) of estimate shrinking from 1.0 to \qty{0.51}{\watt\per\metre\squared} while the uncertainty for ACI has remained at \qty{1.2}{\watt\per\metre\squared}. 


\begin{table}[H]
\caption[Estimates of aerosol radiative forcing from successive IPCC assessment reports]{Estimates of aerosol radiative forcing from successive IPCC assessment reports. Before AR5, values are stratospheric-temperature-adjusted radiative forcing (SARF); all values are ERF for AR5 and later. RF\textsubscript{ari} refers to radiative forcing from aerosol-radiation interactions, and RF\textsubscript{aci} to aerosol-cloud interactions. If given, the first number is the best estimate, and bracketed values range within 5--95\% in \unit{\watt\per\metre\squared}. Table adapted from \citet{bellouinAerosolForcingClimate2011} and \citet{forsterEarthEnergyBudget2021}}.
\label{tab:ipcc-rf-summary}
\centering

\begin{tabular}{lccc}
\toprule
Report & RF\textsubscript{ari} & RF\textsubscript{aci} & Total aerosol forcing \\
(Period)\\
\midrule
AR2/SAR & \num{-0.50} & N/A & N/A \\
(1750--1993) & (\num{-0.25} to \num{-1.00}) &  (\num{-1.5} to 0.0)  \\ 
AR3/TAR & N/A & N/A & N/A \\
(1750--1998) & & (\num{-2.0} to 0.0) \\
AR4 & \num{-0.50}  & \num{-0.70}  & \num{-1.3}  \\
(1750--2005) & (\num{-0.9} to \num{-0.1}) & (\num{-1.8} to \num{-0.3}) & (\num{-2.2} to \num{-0.5}) \\
AR5 & \num{-0.45} & \num{-0.45}  & \num{-0.90}  \\
(1750--2011) &  (\num{-0.95} to \num{+0.05}) & (\num{-1.2} to 0.0) & (\num{-1.9} to \num{-0.1})   \\
AR6 & \num{-0.22}  & \num{-0.84}  & \num{-1.1} \\
(1750--2019) & (\num{-0.47} to \num{+0.04}) & (\num{-1.45} to \num{-0.25}) & (\num{-1.7} to \num{-0.4})  \\
\bottomrule
\end{tabular}
\end{table}

% Reasons for poor constraints on aerosol climate interactions.
Three main reasons contribute to poor constraints on ACI: observational limitations \citep{raschObservationalConstraintCloud2018, johnsonRobustObservationalConstraint2020}, model structural deficiencies \citep{woodImprovingOurFundamental2016, regayreIdentifyingClimateModel2023,rostronRobustObservationalConstraint2019} and the variable and complex nature of aerosols and their interaction with the Earth system \citep{liScatteringAbsorbingAerosols2022, donnerClimateEffectsAerosolCloud2014, malavelleStrongConstraintsAerosol2017}. Observational limitations apply to satellite and \textit{in situ} observations with limited coverage, resolution or sensitivity. While observation is essential for model validation, it is not the focus of this thesis. The following sections examine the complex nature of aerosol formation and modelling of aerosol-climate interaction, which are central to understanding the poor constraints.


\section{Tropospheric aerosols}

\begin{figure}[H]
    \centering
    \includegraphics[width=0.8\linewidth]{Chapter1/figs/aerosol_size_distribution_cao2013.png}
    \caption{Idealised example of an ambient particle size distribution described in \citet{caoEvolutionPM25Measurements2013}. TSP denotes total suspended particles by a high-volume sampler in the article size range up to \qty{50}{\micro\metre}. Nucleation mode aerosol has a diameter up to \qty{0.01}{\micro\metre}. Ultrafine, also known as the Aitken mode aerosol, has a diameter up to \qty{0.1}{\micro\metre}.}
    \label{fig:aerosol-size-dist}
\end{figure}

Atmospheric aerosols can be divided into broad classes according to size: the nucleation mode, around 10 nm, and Aitken mode particles, having radii less than 100 nm, are formed in the early stages of gas-particle transfer and have a short lifetime controlled by loss via diffusion to existing particle surfaces; accumulation mode particles, 100 nm in size and greater, are formed by coalescence of smaller particles and condensational growth, diffuse more slowly and have a longer lifetime; finally, coarse-mode particles, which are usually formed through mechanical processes such as wave-breaking or re-suspension of solid particles such as sand or soil, have a short lifetime due to gravitational settling \citep{seinfeldAtmosphericChemistryPhysics2016}. Figure \ref{fig:aerosol-size-dist} shows an idealised example of aerosol size distribution by mass. It shows the variety of aerosol composition across modes.

In addition to size, aerosols may also be categorised by emission. While some aerosols, such as sea salt, soot (black carbon) and dust, are released directly into the atmosphere primarily as accumulation and coarse mode aerosols, a large percentage of aerosols form from their precursor gases in the atmosphere. The latter type may include aerosols such as sulfate, ammonium, nitrate and organic aerosol and are called secondary aerosol \citep[e.g.,][]{jimenezEvolutionOrganicAerosols2009, bauerTurningPointAerosol2022}. Figure \ref{fig:aerosol-composition} depicts the distribution of aerosol with a diameter greater than \qty{2.5}{\micro\metre} (PM2.5). It shows that secondary aerosol, especially sulfate, contributed to a large proportion of aerosol worldwide.

\begin{figure}[H]
    \centering
    \includegraphics[width=\linewidth]{Chapter1/figs/SRL-image-6.png}
    \caption{Distribution of PM2.5 composition mass concentration (in \unit{\micro\gram\per\cubic\metre}) for the major PM2.5 aerosol components. Those aerosol components are sulphate, nitrate, ammonium, sodium, chloride, organic carbon and elemental carbon \citep{szopaShortlivedClimateForcers2021}.}
    \label{fig:aerosol-composition}
\end{figure}

% insert composition of aerosol and add that it is a significant source of secondary aerosol.
As sulfates are secondary aerosols, skilful model treatment of sulfate direct and indirect aerosol forcing requires models to calculate production, loss, and responses. This thesis quantifies the interactions between sulfate aerosol formation and chemistry by analysing chemical reaction tendencies and relating chemical production to cloud properties and radiative impacts. 

\subsection{Sulfate aerosol emissions}
\label{ch1:so2-oxidation}

\begin{figure}
    \centering
    \includegraphics[width=0.8\linewidth]{Chapter1/figs/sulfur_budget.png}
    \caption[Summary of atmospheric sulfur budget]{Summary of atmospheric sulfur budget}
    \label{fig:sulfur-budget}
\end{figure}


Sulfate aerosols are emitted as primary aerosols and produced as secondary aerosols from chemical reactions in the atmosphere by gas or aqueous phase reactions of \ce{SO2}, dimethyl sulfide (DMS) and carbonyl sulfide \citep[COS; ][]{belvisoAssessmentMarineBiota2000}. Figure \ref{fig:sulfur-budget} summarises the atmospheric sulfur budget, which will be described below. First, the emission of sulfate aerosol precursors, \ce{SO2}, is detailed, followed by the formation of sulfate aerosols.

\ce{SO2} can form as products of precursor gases emitted by biogenic processes in the reduced forms of \ce{H2S}, \ce{(CH3)2S} (dimethyl sulfide; DMS), and \ce{(CH3)2S2} (dimethyl disulfide). DMS is an important sulfur-bearing gas produced by algae in the oceans and is believed to be the source background of \ce{SO2} in the atmosphere. Sulfur-containing gases are oxidized into \ce{SO2} and are considered the secondary emission sources for \ce{SO2}, contributing 15--35 Tg(S) yr$^{-1}$ of \ce{SO2} globally \citep{lanaUpdatedClimatologySurface2011}. 

Anthropogenic activities from industries and fossil fuel combustion are the primary source of \ce{SO2}, contributing 50--70 Tg (S) yr$^{-1}$ \citep{forsterEarthEnergyBudget2021}. In addition to anthropogenic sources, volcanic activities, both continuously degassing and sporadic eruptive volcanoes, account for 2.25  Tg (S) yr$^{-1}$ between 1978 and 2014 \citep{carnMultidecadalSatelliteMeasurements2016}. 

Once released or formed in the atmosphere, \ce{SO2} can react with other gases, dissolve in a liquid droplet to undergo an aqueous reaction, or directly react with aerosol surfaces, ultimately forming sulfate aerosols. 

\ce{SO2} reacts with hydroxyl radicals (OH) in the gas phase. OH is one of the atmosphere's most widespread and reactive radicals. In unpolluted air, OH is produced by the photolysis of ozone (\ce{O3}) and the subsequent reaction of oxygen atoms with water vapour \citep{wayneChemistryAtmospheresIntroduction2006}. In polluted air, the photolysis of nitrous acid (HONO) and hydrogen peroxide (\ce{H2O2}) yields OH directly. 

OH radical reacts rapidly with \ce{SO2} in the atmosphere through the reaction
\begin{align}
\ce{SO2 + OH + M -> HOSO2 + M}    
\label{ch1:eq:so2-oh}
\end{align}

M represents a molecule (usually \ce{N2}) that absorbs excess kinetic energy from the reactants. The free radical \ce{HOSO2} reacts in a chain of reactions with \ce{O2} and \ce{H2O} to form sulfuric acid, \ce{H2SO4}.
\begin{align}
\ce{HOSO2 + O2 &-> HO2 + SO3}\\
\ce{SO3 + H2O &-> H2SO4}
\end{align}

Once formed, \ce{H2SO4} (g) reacts with, e.g., \ce{NH3}, amines, \ce{H2O} and nucleates into new aerosol particles \citep{seinfeldAtmosphericChemistryPhysics2016}.

Alternatively, \ce{SO2} can dissolve into liquid droplets, forming an equilibrium with its ionic products (\ce{SO2.H2O}), bisulfate ions (\ce{HSO3-}) and sulfite ions (\ce{SO3^2-}), collectively with \ce{SO2}, these ions have a +4 oxidation state. These +4 oxidation state sulfur species undergo a chain of reactions to form +6 oxidation state species such as sulfate, \ce{SO4^2-}. 
\begin{align}
    \ce{HSO3- + H2O2 &-> SO4^2-} \\
    \ce{HSO3- + O3 &-> SO4^2-} \\
    \ce{SO3^2- + O3 &-> SO4^2-} 
\end{align}

% Discuss Karset
In essence, sulfate aerosols form from \ce{SO2} reactions with oxidants. That is, the availability of tracer gases impacts sulfate aerosol formation, which may have a downstream effect on the resulting secondary aerosols' interaction with radiation, as addressed by \citet{karsetStrongImpactsAerosol2018}. In addition, \citet{oconnorApportionmentPreIndustrial2022} quantified the effects of \ce{CH4} in modulating the cloud radiative effects through modifying oxidants. Increased \ce{CH4} from the year 1850 to 2014 led to cloud radiative effects of \qty{0.12+-0.02}{\watt\per\metre\squared} from thermodynamic adjustments and aerosol-cloud interactions. In particular, the aerosol formation shifted towards larger aerosol, which decreased cloud droplet number concentration (CDNC). These results emphasised the importance of chemistry-aerosol interactions and their effects on climate forcing. 

Current areas of interest include the seasonal cycle and the equatorward shift in global emissions. Earlier research shows that \ce{SO2} conversion rate to sulfate by \ce{SO2 + OH} doubled in summer compared to winter \citet{meagherSeasonalVariationAtmospheric1983}. A model study also indicates that the oxidant levels and respective oxidation exhibit a seasonal cycle \citet{feichterSimulationTroposphericSulfur1996}. 

In addition to the variation in oxidants. \ce{SO2} emission has been shifting equatorward and eastward in the last century, and its implications and current understanding have not been widely discussed for historical emissions. \citet{zhangTroposphericOzoneChange2016} studied the effect of ozone precursor emission redistribution over the period 1980 to 2010 using climate models. The study separated the change in the magnitude of ozone precursor emission from the shift in emission location. It concluded that the global tropospheric ozone change in this period is mainly due to the shift in the emission region rather than the magnitude of emission. Hence, the atmosphere is sensitive to emission location. Whether aerosol precursor emission migration concurred with ozone precursor migration would lead to the same effect as ozone has not yet been determined. 

\subsection{Sulfate aerosol losses}

The removal of atmospheric trace gases and aerosols can be divided into two broad categories, depending on the involvement of precipitation. Dry deposition refers to the process that removes gas or aerosol from the atmosphere onto the surfaces of the Earth, including oceans \citep{dewysAssessmentFateSulfur1978}. Wet deposition describes the removal of gases and particles by incorporation into precipitation \citep{wayneChemistryAtmospheresIntroduction2006}. 

To undergo dry deposition, the trace gas must come into contact with the Earth's surface, such as tree foliage or the ground, via, e.g., turbulent mixing, and be removed by some specific chemical or biological interaction. Deposition velocities range from a few millimetres per second to a few centimetres \citep[e.g.,][]{smithAirborneTransportSulphur1975, hardacreEvaluationSO2SO422021, mulcahyUKESM11DevelopmentEvaluation2023}.

Wet deposition is significant for gaseous compounds that are water-soluble. Less soluble compounds may react and form a more soluble compound before wet-scavenging \citep{wayneChemistryAtmospheresIntroduction2006}. For example, \ce{SO2} gets oxidised, forming \ce{H2SO4}, before wet scavenging removes it from the atmosphere \citep{seinfeldAtmosphericChemistryPhysics2016}.


\section{Sulfate aerosol and chemistry–climate interactions in global climate models} %Section - 1.2
\label{section1.2}

The interaction between atmospheric chemistry and secondary aerosol formation remains complex and not easily quantifiable, yet, as the previous section demonstrates, these processes need to be addressed concurrently. This section examines the development and implementation of chemical and physical processes in numerical models. It outlines ongoing advancements in the representation of their coupled dynamics, which open up opportunities for this thesis.

\subsection{Development of global sulfur modelling}

The first global circulation model to simulate a complete annual sulfur cycle was developed by \citet{langnerGlobalThreedimensionalModel1991}. It represented three-dimensional transport and chemistry to study the global distribution of sulfur compounds. Earlier studies were limited to two-dimensional latitude–height frameworks \citep{rodheGlobalDistributionSulfur1980} or oversimplified reaction pathways \citep{ericksoniiiGlobalOceantoatmosphereDimethyl1990}. Langner's model included three key sulfur species: DMS from natural sources, \ce{SO2} from natural and anthropogenic emissions, and sulfate aerosols. These underwent gas- and aqueous-phase oxidation with prescribed global or zonal reaction rates to produce sulfate as the end product. Aqueous-phase reactions were connected to prescribed climatological cloud cover, and wet and dry deposition were parameterised in space and time. Transport used 10\textdegree by 10\textdegree horizontal resolution, 10 vertical layers, a one-hour timestep, and included spin-up. Despite simplifications, results broadly matched observed concentrations, particularly in polluted regions.

In the following decade, sulfur chemistry became integrated into general circulation and chemical transport models \citep[e.g.,][]{kochTroposphericSulfurSimulation1999}. The first major intercomparison, the COmparison of large-scale atmospheric Sulfate Aerosol Model or COSAM, evaluated 11 models, three of which had interactive meteorology \citep{barrieComparisonLargescaleAtmospheric2001, lohmannVerticalDistributionsSulfur2001, roelofsAnalysisRegionalBudgets2001}. Results showed that surface concentrations of \ce{SO4} were simulated more accurately than \ce{SO2}, indicating vertical mixing as a key uncertainty \citep{lohmannVerticalDistributionsSulfur2001}. Among these models, the GISS GCM simulated \ce{SO2}, sulfate, \ce{H2O2}, DMS, and MSA, and could estimate sulfate's direct radiative effect. Experiments revealed little difference between online and offline chemistry globally, but up to 10\% variation in polluted regions \citep{kochTroposphericSulfurSimulation1999}.

By 2005, most models used one of two approaches to simulate the sulfur cycle: offline meteorology and online chemistry inside a chemical transport model (CTM) or prescribed aerosol for calculating radiative forcing in a GCM. In their 2005 review, \citet{lohmannGlobalIndirectAerosol2005} stated that global model estimates of aerosol direct and indirect effects were highly uncertain. 

By 2010, \citet{tsaiSulfurCycleSulfate2010} had incorporated tropospheric sulfur chemistry into a global climate model, aiming to estimate the radiative forcing of direct and indirect aerosol effects. The result is interactive tropospheric sulfur chemistry within a global climate model. Other research centres followed this development \citep[See ][]{lamarqueAtmosphericChemistryClimate2013}.

In summary, numerical models have grown more complex since the first global transport-chemistry model developed by \citet{langnerGlobalThreedimensionalModel1991}, increasing interactions and coupling over time to represent the Earth as accurately as possible. Figure \ref{fig:model-dev-history} shows the timeline for including components in coupled climate models. Aerosols are treated interactively in global climate models a decade before fully interactive chemistry was integrated \citep{kochTroposphericSulfurSimulation1999,tsaiSulfurCycleSulfate2010}. 

\begin{figure}[H]
    \centering
    \includegraphics[width=0.8\linewidth]{Chapter1/figs/IPCC_AR5_ch1_Fig1-13.jpg}
    \caption[Development of coupled climate models over the last 50 years.]{The development of climate models over the last 50 years showing how the different components were coupled into comprehensive climate models over time. In each aspect (e.g., the atmosphere, which comprises a wide range of atmospheric processes), the complexity and range of processes have increased over time (illustrated by growing cylinders). The horizontal axis refers to the IPCC report. FAR=First Assesment Report \citep{ipccClimateChange19901992}. SAR=Second Assessment Report  \citep{ipcc1995}. TAR=Third Assessment Report \citep{ipcc2001}. AR4 and AR5 are the Fourth and Fifth Assessment Reports, respectively \citep{ipccClimateChange20072007,ipccClimateChange20132014}. Figure from AR5 \citep{Cubasch2013}.}
    \label{fig:model-dev-history}
\end{figure}

\subsection{Atmospheric chemistry modelling and evaluation in the CMIP6 era}
\label{ch1:sec:current-modelling-cmip6}

The scope of questions to which coupled climate models may be applied has broadened as models represented more processes and interactions in the Earth system \citep{flatoEarthSystemModels2011}. As shown in Figure \ref{fig:model-dev-history}, interactive atmospheric chemistry was included for the first time when the IPCC AR5 was compiled. This section summarises the research development since IPCC AR5, specific to aerosol and atmospheric chemistry.

IPCC assessment reports rely on the scientific community coming together to develop and evaluate coupled climate models. The Coupled Model Intercomparison Project (CMIP) provides a protocol for systematically defining model simulations to be performed with coupled climate models and studying the generated output, a process which serves to facilitate model improvement \citep{eyringOverviewCoupledModel2016}. CMIP serves as a foundational pillar for the IPCC Assessment Reports and has evolved over six phases into a major international multi-model research activity \citep{taylorOverviewCMIP5Experiment2012,eyringOverviewCoupledModel2016}. The CMIP phases and IPCC AR report cycle are synchronous. For example, the latest IPCC AR6 report was backed by activities in CMIP6, and IPCC AR5 was supported by CMIP5\footnote{ FAR preceded CMIP and early model coordination began post-FAR. CMIP2 supported SAR. CMIP3 supported AR4. There was a jump from CMIP3 to CMIP5, skipping CMIP4, due to CMIP3 being a major leap forward, and the experiment design was overhauled in CMIP5, with the addition of formal endorsement for MIPs for the first time \citep{taylorOverviewCMIP5Experiment2012}. }.

As part of CMIP5, the Atmospheric Chemistry and Climate Model Intercomparison Project (ACCMIP) proposed simulations aiming at diagnosing the radiative forcing of short-lived climate forcing agents (SLCFs) such as \ce{O3} precursors, black carbon and sulfate aerosols \citep{lamarqueAtmosphericChemistryClimate2013}. Eight models that participated in ACCMIP included some forms of atmospheric chemistry or aerosol simulation, and provided simulation output for diagnosing aerosol ERF \citep{shindellRadiativeForcingACCMIP2013}. ACCMIP has shown that the uncertainty for total radiative forcing, comprised of well-mixed greenhouse gases, \ce{O3} and aerosol, originated from the uncertainty in the aerosol component.

In the proceeding CMIP, CMIP6, the Aerosol Chemistry Model Intercomparison Project (AerChemMIP) improved upon ACCMIP to address the uncertainty in SLCF radiative forcing \citep{collinsAerChemMIPQuantifyingEffects2017,eyringOverviewCoupledModel2016}. AerChemMIP proposed simulations that aimed to attribute changes to the climate system from individual SLCFs, such as \ce{CH4},  halocarbons, N2O, \ce{O3} precursors, aerosol precursors including \ce{SO2}.

Not only is experiment design aimed at attribution, more coupled climate models featured interactive aerosols and chemistry that participated in CMIP6, with at least nine unique models featured interactive chemistry and aerosol\footnote{BCC-ESM1, CESM2-WACCM, EC-Earth3-AerChem, GFDL-ESM4, GISS-E2-1-G, IPSL-CM5A2-IN, MIROC-ES2H, MRI-ESM2-0, and UKESM1-0-LL featured both interactive chemistry and aerosol, according to \citet{ipccAnnexIIModels2023}, see Appendix \ref{app1:title} for details of each model.} \citep[e.g.,][]{zhangBCCESM1ModelDatasets2021, vannoijeECEarth3AerChemGlobalClimate2021, mulcahyDescriptionEvaluationAerosol2020}. For aerosol processes, there have been some improvements to sulfate aerosol modelling in CMIP6 models compared to CMIP5 \citep{forsterEarthEnergyBudget2021}. Some models include pH dependency of \ce{SO2} oxidation \citep[such as the NorESM1; ][]{kirkevagProductiontaggedAerosolModule2018} and explicit descriptions of ammonium and nitrate aerosol components \citep{bianInvestigationGlobalParticulate2017}. This simulation design has led to a comprehensive intercomparison of ERF due to SLCFs \citep[e.g., ][]{thornhillEffectiveRadiativeForcing2021,griffithsTroposphericOzoneCMIP62021,thornhillClimatedrivenChemistryAerosol2021,wangCompensationCloudFeedback2021,stevensonTrendsGlobalTropospheric2020,smithEffectiveRadiativeForcing2020}.


The interaction between atmospheric chemistry, aerosols, and climate dynamics remains a significant source of uncertainty in Earth system modelling. Notably, several CMIP6 models appear to underestimate global surface air temperature (TAS) between 1960 and 1990, often referred to as the "pothole" period \citep{zhangRoleAnthropogenicAerosols2021, flynnStrongAerosolCooling2023}.  The study by \citet{zhangRoleAnthropogenicAerosols2021} indicated that six models, table \ref{tab:zhang-model}, that participated in the CMIP6 tended to underestimate surface air temperature (TAS) during the pothole period. This period coincided with elevated industrial activities over the Northern Hemisphere, especially in the Western European region. \ce{SO2} emitted by these industrial activities was one of the precursors of atmospheric sulfate aerosols, which cooled down the atmosphere via aerosol direct and indirect effects. However, a broader analysis by \citet{flynnStrongAerosolCooling2023} found no singular explanation for this cooling bias. The study highlighted the variability in aerosol schemes across models, including differing treatments of aerosol–cloud and aerosol–radiation processes, as a possible contributing factor. 

\begin{figure}[H]
    \centering
    \includegraphics[width=0.9\linewidth]{Chapter1/figs/pothole_figure_Zhang2021.png}
    \caption[Anomalous cooling in the Earth system models participating in CMIP6]{(a) Historical near-global mean (60\textdegree S to 60\textdegree N) surface air temperature anomalies relative to 1850--1900 surface air temperature (TAS) from HadCRUT5 (thick black line), the ensemble mean for each ESM (solid colour lines), and multi-model mean (MMM, dashed black line). Panel (b) is the same as panel (a) but for the lower-complexity models. Units: \textdegree C. Value in bracket is the number of available members for each model. Figure from \citep{zhangRoleAnthropogenicAerosols2021}.}
    \label{fig:zhang-pothole}
\end{figure}

With the launch of CMIP7, attention is being directed toward the evolving role of aerosols in the Earth system \citep{dunneEvolvingCoupledModel2024}. These emissions are undergoing substantial changes in chemical composition and regional distribution (as documented in https://github.com/JGCRI/CEDS, journal article describing the data in preparation), posing new challenges for model intercomparison. Recent literature continues to highlight the importance of improving these representations, especially concerning regional dynamics and interactions with clouds and radiation \citep{bauerTurningPointAerosol2022, persadAerosolsMustBe2022, bellouinBoundingGlobalAerosol2020, wilcoxRegionalAerosolModel2022}. In particular, the complexities of aerosol radiative effects remain a persistent source of uncertainty in climate simulations. These insights necessitate a deeper understanding of the chemistry-aerosol-climate nexus.


\section{Thesis structure, aims and research questions}

This thesis focuses on the role of atmospheric chemistry in climate models by analysing the \ce{SO2} oxidation pathways using the state-of-the-art Earth system model, UK Earth System Model version 1 (UKESM1), which features fully interactive atmospheric chemistry and emission-driven aerosol simulations \citep{sellarUKESM1DescriptionEvaluation2019,archibaldDescriptionEvaluationUKCA2020}. With the benefit of interactive chemistry, levels of oxidants that are important for sulfate aerosol formation, including OH, \ce{O3}, \ce{H2O2}, and \ce{SO2}, respond to Earth system changes such as seasonal variability of solar radiation and in SLCF emissions. Identifying the processes and interactions within the climate model will improve future prediction capacity and reduce the uncertainty in aerosol forcing calculation. 

\begin{description}

    \item [Chapter 2] provides an overview of the UKESM1 capabilities and evaluation, and the analysis methodology used across the thesis. 
    
    UKESM1 is one of the most complex models that participated in CMIP6 and its endorsed project, AerChemMIP. This research benefits from the UKESM1 simulations for AerChemMIP and the wider CMIP6.

    \item [Chapter 3] aims to understand sulfate aerosol formation drivers in UKESM1 over the historical period. This chapter explores the global trends of sulfur budget and oxidation tendencies. Over the historical period, emissions of SLCFs, such as \ce{CH4} and \ce{O3} precursors, have changed, impacting the atmospheric trace gases.

    Oxidant level changes have a downstream effect on sulfate aerosol production since \ce{SO2} oxidation produces the majority of sulfate aerosols in the atmosphere. This cascade of interactions may impact aerosol radiative effects. 

    Given this, Chapter 3 investigates the following questions: 

    \begin{enumerate}
        \item  How do oxidant levels respond to changes in SLCF emissions over the historical period?
        \item  How do \ce{SO2} budget and oxidation tendencies respond to changes in oxidant levels over the historical period?
        \item How does sulfate aerosol size distribution respond to oxidant perturbation?
        \item At the process level, how do oxidant changes affect aerosol radiative forcing due to changes in oxidants?
    \end{enumerate}

    \item[Chapter 4] focuses on the seasonal variability of sulfate aerosol formation. Sulfate aerosol formation occurs on a sub-annual basis, with sulfate aerosol lifetime in the order of 7--14 days. This short lifetime poses a question on the impact of seasonal variability of oxidants and cloud droplets on aerosol formation. 

    UKESM1 has the capacity to simulate both aerosol mass and number independently, making it one of the most robust climate models regarding aerosol simulation. This makes UKESM1 a valuable tool for studying the aerosol properties in different seasons.
    s
    As mentioned in the previous section, one of the challenges for ESMs is anomalous cooling between 1960 and 1990, which was attributed to aerosol radiative effects \citep{zhangRoleAnthropogenicAerosols2021}. 

    Given this, Chapter 4 investigates the following questions: 

    \begin{enumerate}[resume]
        \item How do changes in seasonal levels of oxidants affect aerosol formation and subsequent radiative effects?
        \item How does UKESM1 perform regarding simulating GSAT seasonal variability?
    \end{enumerate}

    \item[Chapter 5] explores regional sulfate formation and regional differences due to the chemical background. In the historical period, anthropogenic emissions, including \ce{SO2}, increased with industrial activities in the European region and peaked in 1980 before decreasing. Then, emissions increased sharply in Eastern Asia and peaked in the 2000s. Since \ce{SO2} emission location dictates the background of available precursors and the atmospheric chemistry condition near the equator differs from that of the extratropical region, it is possible that different regions have different baseline oxidation tendencies and also behave differently to oxidant level changes. 
    
    This Chapter further inspects the details in regions, aiming to discover oxidation characteristics in different periods and regions.  
        
    Given this, Chapter 5 investigates the following questions: 
    
   \begin{enumerate}[resume]
    \item What is the effect of emission location change on oxidation tendencies?
    \end{enumerate}

    \item[Chapter 6] concludes this thesis with a summary of the outcomes and areas for future work.
\end{description}

\chapter{Regional aerosol formation in the CMIP6 historical period}
% **************************** Define Graphics Path **************************
\ifpdf
    \graphicspath{{Chapter5/Figs/Raster/}{Chapter3/Figs/PDF/}{Chapter5/Figs/}}
\else
    \graphicspath{{Chapter5/Figs/Vector/}{Chapter5/Figs/}}
\fi

% Multiple model results indicate that the global fast precipitation response to regional aerosol forcing scales with global atmospheric absorption \citep{myhrePDRMIPPrecipitationDriver2017}.

% While the emission region does not impact the scale of temperature response, it may impact the aerosol loading and indirectly control the strength of aerosol forcing.

\section*{Abstracts}
This section focuses on regional aspects of aerosol formation. It deals with both annual and seasonal trends. The region of interest includes Northeastern America, Europe, Eastern Asia and South Asia.

NEA and EUR share similar emission trends where SO2 emissions peaked around 1980 and declined. Whereas, emissions from EAS and SAS are on the increase in the near present. 

\section{Introduction}
A modelling study shows that the production of oxidants over different regions may be different \cite{zhangTroposphericOzoneChange2016}. Another model study evaluates how black carbon aerosol emission at different locations affects ERF differently \citep{williamsStrongControlEffective2022}

Furthermore, geographic location can substantially influence the cooling potential of a given aerosol emission \citep{persadDivergentGlobalscaleTemperature2018}. Relative climate effects of combined sulfate, black carbon, and organic carbon aerosol emissions equivalent to China's total annual emission in 2000 result in different cooling potential.

In addition to temperature impact, which seems to be homogeneous irrespective of the emission region. Multiple model results indicate that the global fast precipitation response to regional aerosol forcing scales with global atmospheric absorption \citep{myhrePDRMIPPrecipitationDriver2017}

This means that the emission region does not impact the scale of temperature response. howeve, the region of emission may impact the aerosol loading.

However, this study investigates from aerosol not emission and here we are interested in the oxidation potential in different region. we focus on regions where the change in emission has been the largest: North America, Europe, India, and China



% From AAmas https://acp.copernicus.org/articles/16/7451/2016/
Emissions metrics have normally been calculated for
global emissions. However, for SLCFs, due to their short
lifetimes compared to large-scale atmospheric mixing times,
and because the chemistry and radiative effects on climate
depends on the regional physical conditions, even the global
mean radiative forcing depends on the region of emissions
(Fuglestvedt et al., 1999; Wild et al., 2001; e.g., Berntsen et
al., 2005; Naik et al., 2005). Then, the emission metric val-
ues will vary for different emission locations (Fuglestvedt
et al., 2010). In addition, distinct patterns in the tempera-
ture response appear from all forcings (Boer and Yu, 2003;
Shindell et al., 2010). A growing literature investigates how
the weights of the emission metrics change as emissions
from different regions of the world are considered. Collins
et al. (2013) assessed variations in emission metrics for four
different regions (East Asia, Europe, North America, and
South Asia) for aerosols and ozone precursors, based on
radiative forcings from consistent multimodel experiments
from the Hemispheric Transport of Air Pollution (HTAP) ex-
periments given by Yu et al. (2013) and Fry et al. (2012).
Collins et al. (2010) also investigated how emission met-
ric values differ between regions, including vegetation re-
sponses. Bond et al. (2011) quantified differences in RFs for
BC and OC emissions from different locations and types of
emissions

\subsection{SO2 emissions trends over different regions}



Explain the emissions trends over NEA, EUR, EAS, SAS. 

The distribution of aerosol by each source is explored in \citet{yangGlobalSourceAttribution2017}. Sulfate aeroosl can be transported further from source area.

\subsection{Oxidant level}

\subsection{Annual cycle of oxidant trends and their drivers for different regions}


